\documentclass[12pt,]{article}
%DIF LATEXDIFF DIFFERENCE FILE


\usepackage{lmodern}
\usepackage{amssymb,amsmath}
\usepackage{ifxetex,ifluatex}
\usepackage{fixltx2e} % provides \textsubscript
\ifnum 0\ifxetex 1\fi\ifluatex 1\fi=0 % if pdftex
  \usepackage[T1]{fontenc}
  \usepackage[utf8]{inputenc}
\else % if luatex or xelatex
  \ifxetex
    \usepackage{mathspec}
  \else
    \usepackage{fontspec}
  \fi
  \defaultfontfeatures{Ligatures=TeX,Scale=MatchLowercase}
\fi
% use upquote if available, for straight quotes in verbatim environments
\IfFileExists{upquote.sty}{\usepackage{upquote}}{}
% use microtype if available
\IfFileExists{microtype.sty}{%
\usepackage{microtype}
\UseMicrotypeSet[protrusion]{basicmath} % disable protrusion for tt fonts
}{}
\usepackage[margin=1in]{geometry}
\usepackage{hyperref}
\hypersetup{unicode=true,
%DIF 27c27
%DIF <             pdftitle={A user--friendly tool to evaluate the effectiveness of no--take marine reserves},
%DIF -------
            pdftitle={A user-friendly tool to evaluate the effectiveness of no-take marine reserves}, %DIF > 
%DIF -------
            pdfborder={0 0 0},
            breaklinks=true}
\urlstyle{same}  % don't use monospace font for urls
\usepackage{graphicx,grffile}
\makeatletter
\def\maxwidth{\ifdim\Gin@nat@width>\linewidth\linewidth\else\Gin@nat@width\fi}
\def\maxheight{\ifdim\Gin@nat@height>\textheight\textheight\else\Gin@nat@height\fi}
\makeatother
% Scale images if necessary, so that they will not overflow the page
% margins by default, and it is still possible to overwrite the defaults
% using explicit options in \includegraphics[width, height, ...]{}
\setkeys{Gin}{width=\maxwidth,height=\maxheight,keepaspectratio}
\IfFileExists{parskip.sty}{%
\usepackage{parskip}
}{% else
\setlength{\parindent}{0pt}
\setlength{\parskip}{6pt plus 2pt minus 1pt}
}
\setlength{\emergencystretch}{3em}  % prevent overfull lines
\providecommand{\tightlist}{%
  \setlength{\itemsep}{0pt}\setlength{\parskip}{0pt}}
\setcounter{secnumdepth}{0}
% Redefines (sub)paragraphs to behave more like sections
\ifx\paragraph\undefined\else
\let\oldparagraph\paragraph
\renewcommand{\paragraph}[1]{\oldparagraph{#1}\mbox{}}
\fi
\ifx\subparagraph\undefined\else
\let\oldsubparagraph\subparagraph
\renewcommand{\subparagraph}[1]{\oldsubparagraph{#1}\mbox{}}
\fi

%%% Use protect on footnotes to avoid problems with footnotes in titles
\let\rmarkdownfootnote\footnote%
\def\footnote{\protect\rmarkdownfootnote}

%%% Change title format to be more compact
\usepackage{titling}

% Create subtitle command for use in maketitle
\newcommand{\subtitle}[1]{
  \posttitle{
    \begin{center}\large#1\end{center}
    }
}

\setlength{\droptitle}{-2em}
  \title{A \DIFdelbegin \DIFdel{user--friendly }\DIFdelend \DIFaddbegin \DIFadd{user-friendly }\DIFaddend tool to evaluate the effectiveness of \DIFdelbegin \DIFdel{no--take }\DIFdelend \DIFaddbegin \DIFadd{no-take }\DIFaddend marine
reserves}
  \pretitle{\vspace{\droptitle}\centering\huge}
  \posttitle{\par}
\subtitle{Marine reserve evaluation}
  \author{}
  \preauthor{}\postauthor{}
  \date{}
  \predate{}\postdate{}

\usepackage{booktabs}
\usepackage{longtable}
\usepackage{array}
\usepackage{multirow}
\usepackage[table]{xcolor}
\usepackage{wrapfig}
\usepackage{float}
\usepackage{colortbl}
\usepackage{pdflscape}
\usepackage{tabu}
\usepackage{threeparttable}
\usepackage{setspace}
\doublespacing
\usepackage{lineno}
\linenumbers
%DIF PREAMBLE EXTENSION ADDED BY LATEXDIFF
%DIF UNDERLINE PREAMBLE %DIF PREAMBLE
\RequirePackage[normalem]{ulem} %DIF PREAMBLE
\RequirePackage{color}\definecolor{RED}{rgb}{1,0,0}\definecolor{BLUE}{rgb}{0,0,1} %DIF PREAMBLE
\providecommand{\DIFaddtex}[1]{{\protect\color{blue}\uwave{#1}}} %DIF PREAMBLE
\providecommand{\DIFdeltex}[1]{{\protect\color{red}\sout{#1}}}                      %DIF PREAMBLE
%DIF SAFE PREAMBLE %DIF PREAMBLE
\providecommand{\DIFaddbegin}{} %DIF PREAMBLE
\providecommand{\DIFaddend}{} %DIF PREAMBLE
\providecommand{\DIFdelbegin}{} %DIF PREAMBLE
\providecommand{\DIFdelend}{} %DIF PREAMBLE
%DIF FLOATSAFE PREAMBLE %DIF PREAMBLE
\providecommand{\DIFaddFL}[1]{\DIFadd{#1}} %DIF PREAMBLE
\providecommand{\DIFdelFL}[1]{\DIFdel{#1}} %DIF PREAMBLE
\providecommand{\DIFaddbeginFL}{} %DIF PREAMBLE
\providecommand{\DIFaddendFL}{} %DIF PREAMBLE
\providecommand{\DIFdelbeginFL}{} %DIF PREAMBLE
\providecommand{\DIFdelendFL}{} %DIF PREAMBLE
%DIF HYPERREF PREAMBLE %DIF PREAMBLE
\providecommand{\DIFadd}[1]{\texorpdfstring{\DIFaddtex{#1}}{#1}} %DIF PREAMBLE
\providecommand{\DIFdel}[1]{\texorpdfstring{\DIFdeltex{#1}}{}} %DIF PREAMBLE
\newcommand{\DIFscaledelfig}{0.5}
%DIF HIGHLIGHTGRAPHICS PREAMBLE %DIF PREAMBLE
\RequirePackage{settobox} %DIF PREAMBLE
\RequirePackage{letltxmacro} %DIF PREAMBLE
\newsavebox{\DIFdelgraphicsbox} %DIF PREAMBLE
\newlength{\DIFdelgraphicswidth} %DIF PREAMBLE
\newlength{\DIFdelgraphicsheight} %DIF PREAMBLE
% store original definition of \includegraphics %DIF PREAMBLE
\LetLtxMacro{\DIFOincludegraphics}{\includegraphics} %DIF PREAMBLE
\newcommand{\DIFaddincludegraphics}[2][]{{\color{blue}\fbox{\DIFOincludegraphics[#1]{#2}}}} %DIF PREAMBLE
\newcommand{\DIFdelincludegraphics}[2][]{% %DIF PREAMBLE
\sbox{\DIFdelgraphicsbox}{\DIFOincludegraphics[#1]{#2}}% %DIF PREAMBLE
\settoboxwidth{\DIFdelgraphicswidth}{\DIFdelgraphicsbox} %DIF PREAMBLE
\settoboxtotalheight{\DIFdelgraphicsheight}{\DIFdelgraphicsbox} %DIF PREAMBLE
\scalebox{\DIFscaledelfig}{% %DIF PREAMBLE
\parbox[b]{\DIFdelgraphicswidth}{\usebox{\DIFdelgraphicsbox}\\[-\baselineskip] \rule{\DIFdelgraphicswidth}{0em}}\llap{\resizebox{\DIFdelgraphicswidth}{\DIFdelgraphicsheight}{% %DIF PREAMBLE
\setlength{\unitlength}{\DIFdelgraphicswidth}% %DIF PREAMBLE
\begin{picture}(1,1)% %DIF PREAMBLE
\thicklines\linethickness{2pt} %DIF PREAMBLE
{\color[rgb]{1,0,0}\put(0,0){\framebox(1,1){}}}% %DIF PREAMBLE
{\color[rgb]{1,0,0}\put(0,0){\line( 1,1){1}}}% %DIF PREAMBLE
{\color[rgb]{1,0,0}\put(0,1){\line(1,-1){1}}}% %DIF PREAMBLE
\end{picture}% %DIF PREAMBLE
}\hspace*{3pt}}} %DIF PREAMBLE
} %DIF PREAMBLE
\LetLtxMacro{\DIFOaddbegin}{\DIFaddbegin} %DIF PREAMBLE
\LetLtxMacro{\DIFOaddend}{\DIFaddend} %DIF PREAMBLE
\LetLtxMacro{\DIFOdelbegin}{\DIFdelbegin} %DIF PREAMBLE
\LetLtxMacro{\DIFOdelend}{\DIFdelend} %DIF PREAMBLE
\DeclareRobustCommand{\DIFaddbegin}{\DIFOaddbegin \let\includegraphics\DIFaddincludegraphics} %DIF PREAMBLE
\DeclareRobustCommand{\DIFaddend}{\DIFOaddend \let\includegraphics\DIFOincludegraphics} %DIF PREAMBLE
\DeclareRobustCommand{\DIFdelbegin}{\DIFOdelbegin \let\includegraphics\DIFdelincludegraphics} %DIF PREAMBLE
\DeclareRobustCommand{\DIFdelend}{\DIFOaddend \let\includegraphics\DIFOincludegraphics} %DIF PREAMBLE
\LetLtxMacro{\DIFOaddbeginFL}{\DIFaddbeginFL} %DIF PREAMBLE
\LetLtxMacro{\DIFOaddendFL}{\DIFaddendFL} %DIF PREAMBLE
\LetLtxMacro{\DIFOdelbeginFL}{\DIFdelbeginFL} %DIF PREAMBLE
\LetLtxMacro{\DIFOdelendFL}{\DIFdelendFL} %DIF PREAMBLE
\DeclareRobustCommand{\DIFaddbeginFL}{\DIFOaddbeginFL \let\includegraphics\DIFaddincludegraphics} %DIF PREAMBLE
\DeclareRobustCommand{\DIFaddendFL}{\DIFOaddendFL \let\includegraphics\DIFOincludegraphics} %DIF PREAMBLE
\DeclareRobustCommand{\DIFdelbeginFL}{\DIFOdelbeginFL \let\includegraphics\DIFdelincludegraphics} %DIF PREAMBLE
\DeclareRobustCommand{\DIFdelendFL}{\DIFOaddendFL \let\includegraphics\DIFOincludegraphics} %DIF PREAMBLE
%DIF END PREAMBLE EXTENSION ADDED BY LATEXDIFF

\begin{document}
\maketitle

Juan Carlos Villaseñor-Derbez\textsuperscript{1¶*}, Caio
Faro\textsuperscript{1¶}, Melaina Wright\textsuperscript{1¶}, Jael
Martínez\textsuperscript{1¶}, Sean Fitzgerald\textsuperscript{1\&},
Stuart Fulton\textsuperscript{2\&}, Maria del Mar
Mancha-Cisneros\textsuperscript{3\&}, Gavin
McDonald\textsuperscript{1,4,5\&}, Fiorenza
Micheli\textsuperscript{6\&}, Alvin Suárez\textsuperscript{2\&}, Jorge
Torre\textsuperscript{2\&}, Christopher Costello\textsuperscript{1,4,5¶}

\textsuperscript{1} Bren School of Environmental Science and Management,
University of California Santa Barbara, Santa Barbara, California,
United States

\textsuperscript{2} Comunidad y Biodiversidad A.C., Calle Isla del
Peruano, Guaymas, Sonora, México

\textsuperscript{3} School of Life Sciences, Arizona State University,
Tempe, Arizona, United States

\textsuperscript{4} Sustainable Fisheries Group, University of
California Santa Barbara, Santa Barbara, California, United States

\textsuperscript{5} Marine Science Institute, University of California
Santa Barbara, Santa Barbara, California, United States

\textsuperscript{6} Hopkins Marine Station and Center for Ocean
Solutions, Stanford University, Pacific Grove, CA 93950, USA

*Corresponding author

Email:
\href{mailto:jvillasenor@bren.ucsb.edu}{\nolinkurl{jvillasenor@bren.ucsb.edu}}
(JCVD)

¶ These authors contributed equally to this work.

\& These authors also contributed equally to this work.

\clearpage

\textbf{Abstract}

Marine reserves are implemented to achieve a variety of objectives, but
are seldom rigorously evaluated to determine whether those objectives
are met. In the rare cases when evaluations do take place, they
typically focus on ecological indicators and ignore other relevant
objectives such as socioeconomics and governance. And regardless of the
objectives, the diversity of locations, monitoring protocols, and
analysis approaches hinder the ability to compare results across case
studies. Moreover, analysis and evaluation of reserves is generally
conducted by outside researchers, not the reserve managers or users,
plausibly thereby hindering effective local management and rapid
response to change. We present a framework and tool, called ``MAREA'',
to overcome these challenges. Its purpose is to evaluate the extent to
which any given reserve has achieved its stated objectives. MAREA
provides specific guidance on data collection and formatting, and then
conducts rigorous causal inference analysis based on data input by the
user, providing \DIFdelbegin \DIFdel{real--time }\DIFdelend \DIFaddbegin \DIFadd{real-time }\DIFaddend outputs about the effectiveness of the
reserve. MAREA's ease of use, standardization of \DIFdelbegin \DIFdel{state--of--the--art
}\DIFdelend \DIFaddbegin \DIFadd{state-of-the-art
}\DIFaddend inference methods, and ability to analyze marine reserve effectiveness
across ecological, socioeconomic, and governance objectives could
dramatically further our understanding and support of effective marine
reserve management.

\section{Introduction}\label{introduction}

Unsustainable fishing practices threaten biodiversity, conservation,
economic and social outcomes {[}1,2{]}. Marine Protected Areas (MPAs;
and marine reserves, in which all extractive efforts are prohibited) are
frequently proposed to aid in the recovery of fish and invertebrate
stocks {[}3--6{]} by limiting or restricting fishing effort and gears.

\DIFdelbegin \DIFdel{Available empirical evidence on marine reserve effectiveness is mixed
}%DIFDELCMD < {[}%%%
\DIFdel{7}%DIFDELCMD < {]}%%%
\DIFdel{. Some studies have shown }\DIFdelend \DIFaddbegin \DIFadd{Empirical evidence shows }\DIFaddend that MPAs increase biomass {[}4,\DIFdelbegin \DIFdel{8}\DIFdelend \DIFaddbegin \DIFadd{7}\DIFaddend {]}, enhance
resilience to climatic impacts {[}\DIFdelbegin \DIFdel{9,10}\DIFdelend \DIFaddbegin \DIFadd{8,9}\DIFaddend {]}, and preserve genetic diversity
{[}\DIFdelbegin \DIFdel{11}\DIFdelend \DIFaddbegin \DIFadd{10}\DIFaddend {]}. Compared to \DIFdelbegin \DIFdel{partially protected MPAs }\DIFdelend \DIFaddbegin \DIFadd{MPAs that grant partial protection}\DIFaddend , marine
reserves have higher levels of biomass, density, richness, and larger
organisms {[}3,\DIFdelbegin \DIFdel{12--14}\DIFdelend \DIFaddbegin \DIFadd{11--13}\DIFaddend {]}. \DIFdelbegin \DIFdel{These }\DIFdelend \DIFaddbegin \DIFadd{However, these }\DIFaddend effects are often measured as
biological changes \DIFdelbegin \DIFdel{in the area through time}\DIFdelend \DIFaddbegin \DIFadd{within the reserves through time, }\DIFaddend and many lack a
control site for comparison {[}\DIFdelbegin \DIFdel{15}\DIFdelend \DIFaddbegin \DIFadd{14}\DIFaddend {]}. This approach does not account for
other factors \DIFaddbegin \DIFadd{(}\emph{\DIFadd{e.g.}} \DIFadd{system-level changes in productivity caused
by predatory release }{[}\DIFadd{15}{]}\DIFadd{; or favorable environmental conditions
}{[}\DIFadd{16}{]}\DIFadd{) }\DIFaddend for which one must control {[}\DIFdelbegin \DIFdel{16}\DIFdelend \DIFaddbegin \DIFadd{17}\DIFaddend {]} in order to causally
attribute a biological change to the reserve. \DIFdelbegin \DIFdel{While some }\DIFdelend \DIFaddbegin \DIFadd{Other }\DIFaddend studies have used \DIFdelbegin \DIFdel{control
sites, these analyses do not estimate the effect of the reserve, and
often use a
control--impact }\DIFdelend \DIFaddbegin \DIFadd{a
control-impact }\DIFaddend comparison approach that \DIFaddbegin \DIFadd{uses control sites but }\DIFaddend does not
address temporal variability {[}4,\DIFdelbegin \DIFdel{8,17--19}\DIFdelend \DIFaddbegin \DIFadd{7,18--20}\DIFaddend {]}.
\DIFaddbegin 

\DIFaddend A smaller fraction of studies have used a \DIFdelbegin \DIFdel{before--after--control--impact }\DIFdelend \DIFaddbegin \DIFadd{before-after-control-impact
}\DIFaddend (\emph{i.e.} BACI) design comparing reserves to control sites before and
after implementation {[}4,\DIFdelbegin \DIFdel{20,}\DIFdelend 21\DIFaddbegin \DIFadd{,22}\DIFaddend {]}, which allows the use of causal
inference techniques that estimate the effect of the reserve. \DIFdelbegin %DIFDELCMD < 

%DIFDELCMD < %%%
\DIFdelend \DIFaddbegin \DIFadd{For
example, in ref }{[}\DIFadd{21}{]} \DIFadd{authors use a BACI design and observe increases
in lobster catches --a proxy for abundances-- after reserve
implementation for protected and control sites. However, the temporal
changes in the reserve were greater than in the control site, suggesting
a positive effect of the reserve on lobster catches. }\DIFaddend But even when
proper causal inference can be drawn, results are often \DIFdelbegin \DIFdel{idiosyncratic }\DIFdelend \DIFaddbegin \DIFadd{different }\DIFaddend across
reserves. Effects of reserves on ecological and economic outcomes are
highly heterogeneous, and often depend on the specific ecological,
economic, and social context.
\DIFdelbegin \DIFdel{The purpose of this
paper is to describe a user--friendly tool, called ``MAREA'', to
rigorously systematize the evaluation of marine reserve effectiveness.
The tool is in the form of an open-source application that uses
state--of--the--art methods from program evaluation to compare a reserve
to control sites along a number of ecological, economic, and governance
dimensions.
}\DIFdelend 

\DIFdelbegin \DIFdel{The challenge of how to standardize }\DIFdelend \DIFaddbegin \DIFadd{Standardization of }\DIFaddend marine reserve evaluation is not \DIFdelbegin \DIFdel{a
newone. The recent }\DIFdelend \DIFaddbegin \DIFadd{new. The }\DIFaddend IUCN
framework ``How is your MPA doing?'' {[}\DIFdelbegin \DIFdel{22,}\DIFdelend 23\DIFaddbegin \DIFadd{,24}\DIFaddend {]} provides a
comprehensive list of biological, socioeconomic, and governance
indicators, and insights into how these \DIFdelbegin \DIFdel{indicators }\DIFdelend may be measured \DIFaddbegin \DIFadd{or collected}\DIFaddend .
But this framework stops short of analysis, \DIFdelbegin \DIFdel{so }\DIFdelend \DIFaddbegin \DIFadd{and }\DIFaddend provides a user with
little guidance about establishing causal inference about the reserve.
Recent work \DIFdelbegin \DIFdel{by Mascia }\emph{\DIFdel{et al.}} %DIFAUXCMD
\DIFdelend {[}\DIFdelbegin \DIFdel{24}\DIFdelend \DIFaddbegin \DIFadd{25}\DIFaddend {]} integrates these three dimensions via the Social
Ecological Systems Framework {[}\DIFdelbegin \DIFdel{25,}\DIFdelend 26\DIFaddbegin \DIFadd{,27}\DIFaddend {]} and suggests the use of causal
inference techniques to provide a measure of the effect of conservation
interventions. However, \DIFdelbegin \DIFdel{these two novel
approaches do not }\DIFdelend \DIFaddbegin \DIFadd{neither of these approaches }\DIFaddend provide a
user-friendly tool that \DIFdelbegin \DIFdel{enables
}\DIFdelend \DIFaddbegin \DIFadd{ensures }\DIFaddend replicability and scalability of the
analysis, particularly when used by the fishers and decision makers
themselves.

An increasingly popular way to make science accessible, reproducible,
scalable, and replicable is through Open Science and the development of
\DIFdelbegin \DIFdel{open--access }\DIFdelend \DIFaddbegin \DIFadd{open-access }\DIFaddend tools {[}\DIFdelbegin \DIFdel{27}\DIFdelend \DIFaddbegin \DIFadd{28}\DIFaddend {]}. The Ocean Health Index {[}\DIFdelbegin \DIFdel{28,}\DIFdelend 29\DIFaddbegin \DIFadd{,30}\DIFaddend {]}, for
example, successfully standardized a way to measure the health and
benefits of the oceans. This approach has been implemented at a global
scale, but also at country-level {[}\DIFdelbegin \DIFdel{30}\DIFdelend \DIFaddbegin \DIFadd{31}\DIFaddend {]}, and regionally {[}\DIFdelbegin \DIFdel{31,}\DIFdelend 32\DIFaddbegin \DIFadd{,33}\DIFaddend {]}.
Open access tools are not limited to conservation, and have also been
developed to evaluate fishery performance {[}\DIFdelbegin \DIFdel{33,}\DIFdelend 34\DIFaddbegin \DIFadd{,35}\DIFaddend {]}, design
territorial use rights for fisheries {[}\DIFdelbegin \DIFdel{35}\DIFdelend \DIFaddbegin \DIFadd{36}\DIFaddend {]}, and improve decision
making in the hydro power industry {[}\DIFdelbegin \DIFdel{36}\DIFdelend \DIFaddbegin \DIFadd{37}\DIFaddend {]}.

\DIFdelbegin \DIFdel{This paper presents a framework and user--friendly toolto evaluate
}\DIFdelend \DIFaddbegin \DIFadd{The purpose of this paper is to describe a user-friendly tool, called
``MAREA'', to rigorously systematize the evaluation of }\DIFaddend marine reserve
effectiveness \DIFdelbegin \DIFdel{, which incorporates the biological}\DIFdelend \DIFaddbegin \DIFadd{in terms of fisheries and marine conservation goals. The
tool is in the form of an open-source application that uses
state-of-the-art methods from program evaluation to compare a reserve to
control sites along a number of biological, economic}\DIFaddend , \DIFdelbegin \DIFdel{socioeconomic, }\DIFdelend and governance
dimensions\DIFdelbegin \DIFdel{of any fishery}\DIFdelend . We first provide a list of commonly stated management
objectives and match them to appropriate indicators. We then develop a
simple approach to analyzing these indicators building on causal
inference techniques {[}\DIFdelbegin \DIFdel{20}\DIFdelend \DIFaddbegin \DIFadd{21}\DIFaddend {]}, which help us understand the effect of
management interventions {[}\DIFdelbegin \DIFdel{24,37}\DIFdelend \DIFaddbegin \DIFadd{25,38}\DIFaddend {]}. To implement the analytical
approach\DIFdelbegin \DIFdel{in a
user-friendly format}\DIFdelend , we introduce the Marine Reserve Evaluation Application
(MAREA), an open source, web--based tool that automates the framework
described in this paper and enables its broader use. Finally, we present
a case study on the evaluation of a marine reserve established by the
fishers of Isla Natividad (Mexico) in 2006, to demonstrate the potential
of MAREA.

\section{Materials and methods}\label{materials-and-methods}

Here, we describe the proposed framework to evaluate the effectiveness
of marine reserves (Fig. 1). We explain how management objectives were
identified and matched to appropriate indicators that allow the
evaluation of the reserves, and provide brief guidelines on data
collection. Alongside, methodologies to analyze these indicators are
presented. We then describe the development of MAREA and explain how
this \DIFdelbegin \DIFdel{user--friendly open--access }\DIFdelend tool can be used by fishermen, managers, and other stakeholders
with little scientific background. Finally, we provide guidelines on how
to interpret and use the results and output generated by MAREA to inform
management.

\textbf{Fig 1. Workflow to evaluate the effectiveness of marine
reserves.}

\subsection{Marine Reserve objectives and
indicators}\label{marine-reserve-objectives-and-indicators}

Throughout this study, we will refer to the stated goals for which a
marine reserve was designed as ``objectives.'' This work was motivated
by the case of Mexico, where 39 reserves have been implemented over the
past five years to achieve objectives such as increasing productivity in
nearby waters or \DIFdelbegin \DIFdel{recover }\DIFdelend \DIFaddbegin \DIFadd{recovery of }\DIFaddend overexploited species; most of these
reserves have never been formally evaluated for effectiveness at meeting
those objectives. Thus, our focus was on identifying common objectives
of marine reserves in Mexico. However, a literature review and
discussions with marine reserve researchers \DIFdelbegin \DIFdel{worldwide suggest }\DIFdelend \DIFaddbegin \DIFadd{suggested }\DIFaddend that the
objectives driving Mexican marine reserve implementation are similar to
those in the rest of the world. Thus, we group these objectives into
seven major categories \DIFdelbegin \DIFdel{, which can }\DIFdelend \DIFaddbegin \DIFadd{that may }\DIFaddend be applied to marine reserves worldwide.
\DIFdelbegin \DIFdel{The list
of objectives includes stated objectives }\DIFdelend \DIFaddbegin \DIFadd{Any given reserve may have been implemented to meet one or more of
these. The list includes objectives stated }\DIFaddend in legislation {[}\DIFdelbegin \DIFdel{38,}\DIFdelend 39\DIFaddbegin \DIFadd{,40}\DIFaddend {]}
and official documents such as the Technical Justification Studies
(\emph{Estudios Técnicos Justificativos}), agreements, and decrees
associated to these areas:

\begin{enumerate}
\def\labelenumi{\arabic{enumi}.}
\tightlist
\item
  Avoid overexploitation
\item
  Conserve species under a special protection regime
\item
  Maintain biological processes (reproduction, recruitment, growth,
  feeding)
\item
  Improve fishery production in adjacent waters
\item
  Preserve biological diversity and the ecosystem
\item
  Recover overexploited species
\item
  Recover species of economic interest
\end{enumerate}

Based on these seven objectives, we determined a set of associated
indicators to evaluate reserve effectiveness. These indicators are
specific variables on which data could be collected \DIFdelbegin \DIFdel{, }\DIFdelend and analyzed, to
ultimately determine whether the corresponding objective was causally
being achieved by the marine reserve. The list of indicators was
compiled through a review of scientific literature in which we
identified indicators that were used to measure similar
objectives\DIFaddbegin {[}\DIFadd{3--5,7,11,13,14,18--21,23,24,41--44}{]}\DIFaddend . A first filter
eliminated indicators for which baseline data do not typically exist in
Mexico. The preliminary list of indicators was reviewed at a workshop
with participation of members from Mexican fishery management agencies
and non-government organizations. Later, these were presented to fishers
from the Ensenada Fishing Cooperative (\emph{S.C.P.P. Ensenada}), in El
Rosario, Baja California, who provided input. Our final list of
indicators includes those identified in review works {[}4,\DIFdelbegin \DIFdel{40}\DIFdelend \DIFaddbegin \DIFadd{44}\DIFaddend {]}.

Indicators are divided into three main categories: biological,
socioeconomic, and governance (Table 1). The nine biological indicators
focus on fish and invertebrate communities that are evaluated using
underwater ecological surveys performed inside and outside the reserve
(see Data and Analysis section for specific sampling design and
methodologies). Five socioeconomic indicators reflect the performance of
the fishery in terms of landings, income from landings, and availability
of alternative livelihoods. Fifteen governance indicators describe the
governance structures under which the community operates (\emph{e.g.},
access rights to the fishery, number of fishers, legal recognition of
the reserve). \DIFdelbegin \DIFdel{Some indicators }\DIFdelend \DIFaddbegin \DIFadd{Most biological and socioeconomic indicators are
quantitative and }\DIFaddend require a numerical entry (\emph{e.g.} Fish biomass)
while \DIFdelbegin \DIFdel{others are more descriptive }\DIFdelend \DIFaddbegin \DIFadd{all governance indicators, one biological indicator, and one
socioeconomic indicator are qualitative and rely on a descriptive entry
}\DIFaddend (\emph{e.g.} Reasoning for reserve location). Many of them specifically
measure an outcome of the reserve, though some are designed to further
the understanding of the mechanisms driving a reserve's performance. In
that sense, most biological and socioeconomic indicators are outcome
variables. On the other hand, governance indicators are viewed as
possible explanatory variables of reserve performance. Whenever an
indicator is applied to ``Target species'', it means that the indicator
can be used for all species (\emph{e.g.} Fish Biomass) and/or for
individual species that are either the conservation target of the
reserve or are of particular economic or ecological interest
(\emph{e.g.} Grouper Biomass). \DIFaddbegin \DIFadd{Finally, indicators B3 and B4 are
different in that B3 only looks at the density of organisms above size
at first maturity (related to reproductive potential), while B4 measures
the density of all fish or of a target species. Each indicator targets
different plausible desired outcomes, like increased reproductive
potential (}\emph{\DIFadd{i.e.}} \DIFadd{B3; }{[}\DIFadd{45}{]}\DIFadd{) or having more fish -regardless of
their size- to attract tourism (}\emph{\DIFadd{i.e.}} \DIFadd{B4). }\DIFaddend Table 1 presents the
proposed indicators, and Table 2 shows how objectives are matched with
biological and socioeconomic indicators. \DIFaddbegin \DIFadd{Governance indicators are
excluded from Table 2, but should be considered for every objective as
each serves as a plausible explanatory variable for reserve performance.
}\DIFaddend 

\DIFdelbegin %DIFDELCMD < \begin{table}
%DIFDELCMD < %%%
\DIFdelendFL \DIFaddbeginFL \clearpage
\DIFaddendFL 

\DIFdelbeginFL %DIFDELCMD < \caption{%
{%DIFAUXCMD
%DIFDELCMD < \label{tab:unnamed-chunk-2}%%%
\DIFdelFL{List of indicators to evaluate the effectiveness of no-take marine reserves.}}
%DIFAUXCMD
\DIFdelendFL \DIFaddbeginFL \textbf{\DIFaddFL{Table 1. List of indicators to evaluate the effectiveness of
no-take marine reserves.}}

\begin{table}[H]
\DIFaddendFL \centering
\DIFdelbeginFL %DIFDELCMD < \resizebox{\linewidth}{!}{\begin{tabular}[t]{l|l|l|l}
%DIFDELCMD < \hline
%DIFDELCMD < Code & Indicator & Data type & Unit\\
%DIFDELCMD < \hline
%DIFDELCMD < \multicolumn{4}{l}{\textbf{Biological}}\\
%DIFDELCMD < \hline
%DIFDELCMD < \hspace{1em}B1 & Shannon diversity index & Continuous & \\
%DIFDELCMD < \hline
%DIFDELCMD < \hspace{1em}B2 & Species richness & Discrete & Number of species/transect\\
%DIFDELCMD < \hline
%DIFDELCMD < \hspace{1em}B3 & Density of mature organisms & Continuous & Percent points\\
%DIFDELCMD < \hline
%DIFDELCMD < \hspace{1em}B4 & Density* & Continuous & Organisms/transect\\
%DIFDELCMD < \hline
%DIFDELCMD < \hspace{1em}B5 & Natural Disturbance & Descriptive & \\
%DIFDELCMD < \hline
%DIFDELCMD < \hspace{1em}B6 & Mean Trophic Level & Continuous & \\
%DIFDELCMD < \hline
%DIFDELCMD < \hspace{1em}B7 & Biomass* & Continuous & kg/transect\\
%DIFDELCMD < \hline
%DIFDELCMD < \multicolumn{4}{l}{\textbf{Socioeconomic}}\\
%DIFDELCMD < \hline
%DIFDELCMD < \hspace{1em}S1 & Total landings* & Continuous & kg\\
%DIFDELCMD < \hline
%DIFDELCMD < \hspace{1em}S2 & Income from total landings* & Continuous & \$\\
%DIFDELCMD < \hline
%DIFDELCMD < \hspace{1em}S3 & Alternative economic opportunities & Ordinal & \\
%DIFDELCMD < \hline
%DIFDELCMD < \multicolumn{4}{l}{\textbf{Governance}}\\
%DIFDELCMD < \hline
%DIFDELCMD < \hspace{1em}G1 & Access to the fishery & Categorical & \\
%DIFDELCMD < \hline
%DIFDELCMD < \hspace{1em}G2 & Number of fishers & Discrete & \\
%DIFDELCMD < \hline
%DIFDELCMD < \hspace{1em}G3 & Legal recognition of reserve & Binary & \\
%DIFDELCMD < \hline
%DIFDELCMD < \hspace{1em}G4 & Reserve type & Descriptive & \\
%DIFDELCMD < \hline
%DIFDELCMD < \hspace{1em}G5 & Illegal harvesting & Ordinal & \\
%DIFDELCMD < \hline
%DIFDELCMD < \hspace{1em}G6 & Management plan & Binary & \\
%DIFDELCMD < \hline
%DIFDELCMD < \hspace{1em}G7 & Reserve enforcement & Descriptive & \\
%DIFDELCMD < \hline
%DIFDELCMD < \hspace{1em}G8 & Size of reserve & Discrete & \\
%DIFDELCMD < \hline
%DIFDELCMD < \hspace{1em}G9 & Reasoning for reserve location & Descriptive & \\
%DIFDELCMD < \hline
%DIFDELCMD < \hspace{1em}G10 & Membership to fisher organizations & Binary & \\
%DIFDELCMD < \hline
%DIFDELCMD < \hspace{1em}G11 & Type of fisheries organizations & Categorical & \\
%DIFDELCMD < \hline
%DIFDELCMD < \hspace{1em}G12 & Representation & Ordinal & \\
%DIFDELCMD < \hline
%DIFDELCMD < \hspace{1em}G13 & Internal Regulation & Binary & \\
%DIFDELCMD < \hline
%DIFDELCMD < \hspace{1em}G14 & Perceived Effectiveness & Categorical & \\
%DIFDELCMD < \hline
%DIFDELCMD < \hspace{1em}G15 & Social Impact of Reserve & Categorical & \\
%DIFDELCMD < \hline
%DIFDELCMD < \multicolumn{4}{l}{\textsuperscript{*} The indicator is also applied to target species}\\
%DIFDELCMD < \end{tabular}}
%DIFDELCMD < %%%
\DIFdelendFL \DIFaddbeginFL \resizebox{\linewidth}{!}{\begin{tabular}{l|l|l|l}
\hline
\bfseries{Code} & \bfseries{Indicator} & \bfseries{Data type} & \bfseries{Unit}\\
\hline
\multicolumn{4}{l}{\textbf{Biological}}\\
\hline
\hspace{1em}B1 & Shannon diversity index & Continuous & \\
\hline
\hspace{1em}B2 & Species richness & Discrete & Number of species/transect\\
\hline
\hspace{1em}B3 & Density of mature organisms & Continuous & Percent\\
\hline
\hspace{1em}B4 & Density* & Continuous & Organisms/transect\\
\hline
\hspace{1em}B5 & Natural Disturbance & Descriptive & \\
\hline
\hspace{1em}B6 & Mean Trophic Level & Continuous & \\
\hline
\hspace{1em}B7 & Biomass* & Continuous & kg/transect\\
\hline
\multicolumn{4}{l}{\textbf{Socioeconomic}}\\
\hline
\hspace{1em}S1 & Total landings* & Continuous & kg\\
\hline
\hspace{1em}S2 & Income from total landings* & Continuous & \$\\
\hline
\hspace{1em}S3 & Alternative economic opportunities & Ordinal & \\
\hline
\multicolumn{4}{l}{\textbf{Governance}}\\
\hline
\hspace{1em}G1 & Access to the fishery & Categorical & \\
\hline
\hspace{1em}G2 & Number of fishers & Discrete & \\
\hline
\hspace{1em}G3 & Legal recognition of reserve & Binary & \\
\hline
\hspace{1em}G4 & Reserve type & Descriptive & \\
\hline
\hspace{1em}G5 & Illegal harvesting & Ordinal & \\
\hline
\hspace{1em}G6 & Management plan & Binary & \\
\hline
\hspace{1em}G7 & Reserve enforcement & Descriptive & \\
\hline
\hspace{1em}G8 & Size of reserve & Discrete & \\
\hline
\hspace{1em}G9 & Reasoning for reserve location & Descriptive & \\
\hline
\hspace{1em}G10 & Membership to fisher organizations & Binary & \\
\hline
\hspace{1em}G11 & Type of fisheries organizations & Categorical & \\
\hline
\hspace{1em}G12 & Representation & Ordinal & \\
\hline
\hspace{1em}G13 & Internal Regulation & Binary & \\
\hline
\hspace{1em}G14 & Perceived Effectiveness & Categorical & \\
\hline
\hspace{1em}G15 & Social Impact of Reserve & Categorical & \\
\hline
\end{tabular}}
\DIFaddendFL \end{table}

\DIFaddbegin \DIFadd{* Indicates the indicator is applied to target species
}

\DIFaddend \clearpage

\DIFdelbegin %DIFDELCMD < \begin{table}
%DIFDELCMD < %%%
\DIFdelendFL \DIFaddbeginFL \textbf{\DIFaddFL{Table 2. Management objectives and respective performance
indicators.}}
\DIFaddendFL 

\DIFdelbeginFL %DIFDELCMD < \caption{%
{%DIFAUXCMD
%DIFDELCMD < \label{tab:unnamed-chunk-3}%%%
\DIFdelFL{Management objectives and respective performance indicators. All governance indicators should allways be used.}}
%DIFAUXCMD
\DIFdelendFL \DIFaddbeginFL \begin{table}[H]
\DIFaddendFL \centering
\DIFdelbeginFL %DIFDELCMD < \resizebox{\linewidth}{!}{\begin{tabular}[t]{>{\raggedright\arraybackslash}p{4cm}|l|l|l|l|l|l|l|l|l|l|l|l|l|l}
%DIFDELCMD < \hline
%DIFDELCMD < Objective & B1 & B2 & B3 & B4 & B4* & B5 & B6 & B7 & B7* & S1 & S1* & S2 & S2* & S3\\
%DIFDELCMD < \hline
%DIFDELCMD < Avoid overexploitation &  &  & x &  & x & x &  &  & x & x & x & x & x & x\\
%DIFDELCMD < \hline
%DIFDELCMD < Conserve species under a special protection &  &  & x &  & x & x &  &  & x & x &  & x &  & x\\
%DIFDELCMD < \hline
%DIFDELCMD < Maintain biological process &  &  &  & x & x & x &  & x & x & x & x & x & x & x\\
%DIFDELCMD < \hline
%DIFDELCMD < Improve fishery production in nearby waters &  &  & x & x & x & x & x & x & x & x & x & x & x & x\\
%DIFDELCMD < \hline
%DIFDELCMD < Preserve biological diversity and the ecosystem &  &  & x &  & x & x &  &  & x &  & x &  & x & x\\
%DIFDELCMD < \hline
%DIFDELCMD < Recover overexploited species & x & x &  & x &  & x & x & x &  &  &  &  &  & x\\
%DIFDELCMD < \hline
%DIFDELCMD < Recover species of economic interest & x & x &  & x &  & x & x & x &  &  &  &  &  & x\\
%DIFDELCMD < \hline
%DIFDELCMD < \multicolumn{15}{l}{\textsuperscript{*} The indicator is applied to target species}\\
%DIFDELCMD < \end{tabular}}
%DIFDELCMD < %%%
\DIFdelendFL \DIFaddbeginFL \resizebox{\linewidth}{!}{\begin{tabular}{>{\raggedright\arraybackslash}p{4cm}|l|l|l|l|l|l|l|l|l|l|l|l|l|l}
\hline
\bfseries{Objective} & \bfseries{B1} & \bfseries{B2} & \bfseries{B3} & \bfseries{B4} & \bfseries{B4*} & \bfseries{B5} & \bfseries{B6} & \bfseries{B7} & \bfseries{B7*} & \bfseries{S1} & \bfseries{S1*} & \bfseries{S2} & \bfseries{S2*} & \bfseries{S3}\\
\hline
Avoid overexploitation &  &  & x & x & x & x & x & x & x & x & x & x & x & x\\
\hline
Conserve species under a special protection &  &  & x &  & x & x &  &  & x & x &  & x &  & x\\
\hline
Maintain biological process & x & x &  & x &  & x & x & x &  &  &  &  &  & x\\
\hline
Improve fishery production in nearby waters &  &  &  & x & x & x &  & x & x & x & x & x & x & x\\
\hline
Preserve biological diversity and the ecosystem & x & x &  & x &  & x & x & x &  &  &  &  &  & x\\
\hline
Recover overexploited species &  &  & x &  & x & x &  &  & x &  & x &  & x & x\\
\hline
Recover species of economic interest &  &  & x &  & x & x &  &  & x &  & x &  & x & x\\
\hline
\end{tabular}}
\DIFaddendFL \end{table}

\DIFaddbegin \DIFadd{Governance indicators are excluded from the table, but all should be
used for any objective.
}

\DIFadd{* Indicates the indicator is applied to target species
}

\DIFaddend \subsection{Data and analyses}\label{data-and-analyses}

In many coastal marine reserves\DIFdelbegin \DIFdel{of Mexico}\DIFdelend , biological data are \DIFaddbegin \DIFadd{often }\DIFaddend collected via
underwater visual censuses as part of a reserve's monitoring program.
Scientific divers \DIFdelbegin \DIFdel{(which are often local fishermen with
guidance from Civil Society Organizations; CSOs) }\DIFdelend record fish and invertebrate richness and abundances,
as well as fish total length along belt transects. Ecological surveys
are typically performed annually in each reserve and corresponding
control site(s), before and after the implementation of the reserve,
providing a sampling design that can be used to draw causal inference.
Control sites are areas where habitat is similar to that of the reserve,
but with presence of fishing activity; in principle these are areas that
are otherwise observationally identical to the reserve site, but where,
for presumably random reasons, a reserve was not implemented. While
transect dimensions (\emph{i.e.} length and width) and sampling methods
might vary from study to study, the general idea remains the same:
richness, abundances, and sizes of organisms are recorded in a
study--specific standardized way. For this reason, MAREA does not assume
specific transect dimensions, and pertinent indicators are calculated
per transect (Table 1). \DIFaddbegin \DIFadd{More information on data collection and
formatting is provided in a guidebook }{[}\DIFadd{46}{]}\DIFadd{, which is available in
English and Spanish in MAREA's interface.
}\DIFaddend 

This sampling design for biological data allows us to use causal
inference techniques {[}\DIFdelbegin \DIFdel{20,41}\DIFdelend \DIFaddbegin \DIFadd{21,47}\DIFaddend {]} to evaluate the effect of the reserve
on biological indicators. The hypothesis that the indicators will
respond to implementation of the reserve is tested by analyzing spatial
and temporal changes in each numeric biological indicator \DIFdelbegin \DIFdel{(all but B5)
}\DIFdelend using
generalized linear models {[}\DIFdelbegin \DIFdel{20}\DIFdelend \DIFaddbegin \DIFadd{21}\DIFaddend {]}. To account for variations in the
environment and survey conditions, covariates that are gathered during
the underwater ecological surveys are included in the
difference-in-differences model with form:

\begin{equation}I_{i,t,z}=\beta_0 + \sum_{t = 2}^T\gamma_{t}Y_t + \beta_1Z_{i,z} + \beta_2P_{i,t}\times Z_{i,z} + \beta_3T_{i,t,z} + \beta_4V_{i,t,z} + \beta_5D_{i,t,z} + \epsilon_{i,t,z}\end{equation}

In this model, \(i\), \(t\), and \(z\) are indices for transect, time,
and zone (control or reserve site), respectively. This model allows us
to estimate the change in an indicator (\(I\)) based on the year
(\(Y\)), a dummy variable that indicates treatment (\(Z\); \emph{i.e.}
control or reserve), an interaction between a dummy variable that
indicates \DIFdelbegin \DIFdel{pre-- or post--implementation }\DIFdelend \DIFaddbegin \DIFadd{before or after implementation }\DIFaddend (\(P\)) and treatment (\(Z\)),
and covariates such as bottom temperature (\(T\); in °C), horizontal
visibility during the survey (\(V\); in m), and depth at which survey
was performed (\(D\); in m). \(\epsilon\) represents the error term
associated to the regression. Here, years are modeled as factors, using
the first year as the reference level. This does not impose a linear
structure in the way an indicator changes through time (\emph{i.e.} the
change in biomass between 2006 and 2007 does not have to be the same as
the change between 2015 and 2016). The treatment and implementation
variables, modeled as dummy variables, are coded as Control = 0 and
Reserve = 1; and \DIFdelbegin \DIFdel{Pre--implementation }\DIFdelend \DIFaddbegin \DIFadd{Before implementation }\DIFaddend = 0 and \DIFdelbegin \DIFdel{Post--implementation }\DIFdelend \DIFaddbegin \DIFadd{After implementation }\DIFaddend = 1,
respectively.

Socioeconomic data are often collected by fishers, natural resource
management agencies, or \DIFdelbegin \DIFdel{CSOs }\DIFdelend \DIFaddbegin \DIFadd{Civil Society Organizations (CSOs) }\DIFaddend by recording
landings, income, and sometimes prices for each species. To control for
inflation, income is adjusted with the country's consumer price index
{[}\DIFdelbegin \DIFdel{42}\DIFdelend \DIFaddbegin \DIFadd{48}\DIFaddend {]}:

\begin{equation}I_t = RI\times \frac{CPI_t}{CPI_T}\end{equation}

Where \(I_t\) represents the adjusted income for year \(t\) as the
product between the reported income for that year and the ratio between
the consumer price index (\(CPI\)) in that year to the most recent
year's (\(T\)) CPI. Since no control sites are typically available for
this data type, numeric socioeconomic indicators (S1 and S2) are
evaluated with a simplified version of eq. 1:

\begin{equation}I_{t}=\beta_0 + \beta_1P_{t} + \epsilon_{t}\end{equation}

While this model does not formally allow for causal inference, we can
still measure changes in mean landings and income before and after the
implementation of the reserve and provide valuable input. For both
models (eq. 1 and eq. 3), we estimate the model \DIFdelbegin \DIFdel{coeficients }\DIFdelend \DIFaddbegin \DIFadd{coefficients }\DIFaddend with
ordinary least squares, and calculate heteroskedastic--robust standard
errors \DIFaddbegin {[}\DIFadd{49}{]}\DIFaddend .

While biological and some economic data are regularly collected,
governance data are typically not available nor systematically collected
by the community or other organizations. Therefore, we created a survey
specifically designed to collect information needed for the proposed
indicators (B5, S3, and G1--G15). The survey is included as
supplementary material in English (S1 Appendix) and Spanish (S2
Appendix). To analyze governance information, we developed a framework
based on a literature review of common governance structures and their
relation to effectiveness in managing fisheries or marine reserves (S3
Table). This approach has been proven to successfully evaluate
governance structures {[}\DIFdelbegin \DIFdel{43}\DIFdelend \DIFaddbegin \DIFadd{50}\DIFaddend {]}. Unlike with biological and socioeconomic
objectives (see eqs 1 and 3), MAREA does not quantitatively analyze
governance information. Rather, it is presented along with the
biological and socioeconomic indicators to provide managers and users
with a more complete description of the reserve.

\subsection{Marine Reserve Evaluation App
(MAREA)}\label{marine-reserve-evaluation-app-marea}

We developed \DIFdelbegin \DIFdel{MAREAin }\DIFdelend \DIFaddbegin \DIFadd{MAREA in }\DIFaddend R version 3.4.2 and R Studio 1.1.383 {[}\DIFdelbegin \DIFdel{44}\DIFdelend \DIFaddbegin \DIFadd{51}\DIFaddend {]}
using the Shiny package {[}\DIFdelbegin \DIFdel{45}\DIFdelend \DIFaddbegin \DIFadd{52}\DIFaddend {]}, to build an interactive web
application hosted on an open server; the MAREA app can be accessed at
\href{turfeffect.shinyapps.io/marea/}{turfeffect.shinyapps.io/marea}.
While the original version was developed in Spanish because it was aimed
for Mexico and other Latin-American countries, all of its content can be
translated by a translation widget available within the app.

MAREA is designed as a \DIFdelbegin \DIFdel{6--step }\DIFdelend \DIFaddbegin \DIFadd{6-step }\DIFaddend process, divided in tabs that appear upon
launching the app. The first tab introduces the app and summarizes the
evaluation process. Then, the user selects management objectives, which
MAREA automatically matches to appropriate indicators, based on Table 2.
Users can also \DIFdelbegin \DIFdel{manually modify selected }\DIFdelend \DIFaddbegin \DIFadd{select and deselect }\DIFaddend indicators based on their interests
and data availability \DIFaddbegin \DIFadd{by ``clicking'' on the check-boxes in MAREA}\DIFaddend . The
user can then load data on one or more reserves, using standard *.csv
text files; sample datasets are provided within MAREA. Once data have
been loaded, MAREA identifies all reserves in the data, and lets the
user select the reserve to be evaluated. At this point, the user can
also specify the year of implementation of the reserve, reserve
dimensions, and indicate target species that are of particular
management interest. MAREA provides the user with a section to confirm
that all the decisions made leading up to that point are correct. Once
the user has confirmed all input data, objectives, and other
information, MAREA performs the formal program evaluation analyses
discussed above. For a typical data set, the automated analysis step
takes less than one second. Finally, the user is taken to the results
tab where all results are presented in a simple format. The user can
also download a more comprehensive technical report produced in *.pdf
format.

The first output is a color--coded scorecard intended to provide a
general overview of the effectiveness of the reserve. The scorecard
provides a global score for the reserve, a general score for each
category of indicators, and an individual score for each indicator. The
global and category--level scores are determined by the percentage of
positive indicators, overall and for each category, respectively. For
numeric biological indicators\DIFdelbegin \DIFdel{(all but B5)}\DIFdelend , the color is defined by the sign of the
interaction term coefficient (\(\beta_2\)) in eq. 1. For socioeconomic
indicators, colors are assigned based on the direction of the slope
(\(\beta_1\)) in eq. 3. Red, yellow, and green are used for
\(\beta_i<0\), \(\beta_i = 0\), and \(\beta_i>0\), respectively. The
intensity of the color is defined by the significance of the
coefficient, testing the null hypothesis of no change (\emph{i.e.}
\(H_0: \beta_i = 0\)) with a Student's t-test. Cutoff values are
\(p < 0.05\) and \(p < 0.1\). Thus, even in a case where
\(\beta_i > 0\), if the coefficient is not significant by standard
measures (\emph{i.e. } \(p>0.1\)), the indicator will be assigned a
yellow color. A legend (Fig. 2) is provided within the scorecard to aid
in the interpretation of these results. Governance indicators are
represented simply by red or green. The color is defined based on what
literature shows to be a negative (red) or positive (green) factor for a
reserve (S3 Table). For example, if the perceived degree of illegal
fishing is high, this indicator will be assigned a red color. However,
due to the nature of some governance indicators, which require the user
to provide a narrative, only some indicators are presented in the
scorecard (although all are included in the technical report).

\textbf{Fig 2. Legend used to interpret the scorecard produced by
MAREA.} Colors indicate direction of change (red = negative; green =
positive), and color intensity is given by the statistical significance.

The second output from MAREA is a technical report intended to
communicate information and statistical results in a more comprehensive
and technical way. This report also includes a scorecard as a summary of
the results, but provides more information for each indicator. For all
numeric biological indicators, the report includes a graph of the value
of the indicator in the reserve and control sites through time. It also
provides a regression table that summarizes the value of all
coefficients in the regression and their respective robust standard
errors. The summary table also provides information on model fit
(\(R^2\)) and significance of the regression.

The scorecard is produced with functions from the Shinydashboard package
{[}\DIFdelbegin \DIFdel{46}\DIFdelend \DIFaddbegin \DIFadd{53}\DIFaddend {]}. The technical report is produced by a parameterized Rmarkdown
document {[}\DIFdelbegin \DIFdel{47}\DIFdelend \DIFaddbegin \DIFadd{54}\DIFaddend {]} processed by the knitr package {[}\DIFdelbegin \DIFdel{48}\DIFdelend \DIFaddbegin \DIFadd{55}\DIFaddend {]}. Another
feature of MAREA is that the user can choose to share the data. Once the
technical report is downloaded, the information on the reserve, its
management objectives, and all uploaded data are saved into a central
repository. These data can be accessed at any time by any person
interested in acquiring them at \url{github.com/turfeffect/MAREAdata}.

\subsection{Case study}\label{case-study}

While MAREA is a general tool that can be easily employed to evaluate
the effectiveness of any marine reserve with the required input data, we
illustrate its use here by applying it to one marine reserve near Isla
Natividad, in Baja California Sur, Mexico. Isla Natividad is located 8
Km off the Pacific Coast of the Baja California Peninsula (Fig. 3),
where fishers operate under a fishing cooperative (\emph{S.C.P.P. Buzos
y Pescadores de la Baja California}) that promotes co-management of
marine resources {[}\DIFdelbegin \DIFdel{49,50}\DIFdelend \DIFaddbegin \DIFadd{56,57}\DIFaddend {]}. Additionally, fishers have Territorial Use
Rights for Fisheries (TURFs) that provide them with exclusive access
rights to exploit the benthic marine resources within a given perimeter
{[}\DIFdelbegin \DIFdel{50}\DIFdelend \DIFaddbegin \DIFadd{57}\DIFaddend {]}.

\textbf{Fig 3. General location of Isla Natividad (left) and map of the
island (right).} The marine reserve polygon is indicated in red, and the
approximate location of control sites is indicated by blue squares (B =
Babencho, D = La Dulce). \DIFaddbegin \DIFadd{Shapefiles for Mexican coastline and the United
States were obtained from INEGI }{[}\DIFadd{58}{]} \DIFadd{and the tmap R package
}{[}\DIFadd{59}{]}\DIFadd{, respectively.
}\DIFaddend 

In 2006, the Isla Natividad community \DIFdelbegin \DIFdel{implemented two community--based
}\DIFdelend \DIFaddbegin \DIFadd{established a biological baseline
following the data collection protocol described in this study. The
community then implemented two community-based }\DIFaddend marine reserves within
their TURF {[}\DIFdelbegin \DIFdel{9,51,52}\DIFdelend \DIFaddbegin \DIFadd{8,41,60}\DIFaddend {]} \DIFaddbegin \DIFadd{after establishing a baseline for the
soon-to-be reserves and control sites}\DIFaddend . Evidence suggest that these
reserves have been effective at enhancing resilience to climate
variations {[}\DIFdelbegin \DIFdel{9}\DIFdelend \DIFaddbegin \DIFadd{8}\DIFaddend {]} and preserving genetic diversity of high value
commercial species such as abalone {[}\DIFdelbegin \DIFdel{11}\DIFdelend \DIFaddbegin \DIFadd{10}\DIFaddend {]}. These ecological benefits
have been translated into economic benefits, enhancing population
persistence and bolstering abalone fisheries {[}\DIFdelbegin \DIFdel{53}\DIFdelend \DIFaddbegin \DIFadd{43}\DIFaddend {]}. For the purpose
of this evaluation, we focused on the ``La Plana / Las Cuevas'' marine
reserve, located at the southern end of the island (Fig. 3) and its
corresponding control site ``La Dulce / Babencho''.

The objective of this reserve was to recover species of economic
interest ----which were overexploited-- and to enhance fishery
production in nearby waters. Fishers were also interested in preserving
biological diversity and the ecosystem. Thus, objectives 4---7 were
selected. Using Table 2 to match these objectives with appropriate
management indicators, we selected all biological, socioeconomic, and
governance indicators included as options in the framework.

Local fishers (who were trained in scientific diving by the CSO
Comunidad y Biodiversidad, A.C. (COBI; \url{www.cobi.org}), ReefCheck
California, and Stanford University) and personnel from these
institutions performed SCUBA dives to record fish and invertebrate
richness and abundances, as well as fish total length. They recorded
information along 30 m transects, with a sampling window of 2 m x 2 m
following a standardized ReefCheck protocol {[}\DIFdelbegin \DIFdel{54}\DIFdelend \DIFaddbegin \DIFadd{61}\DIFaddend {]}. Ecological surveys
were performed yearly in each reserve and corresponding control site(s),
before and after the implementation of the reserve, providing the
requisite time series data inside the reserve and for a suitable control
site. Annual surveys (2006--2016) were carried out in late July -- early
August, performing a total of 242 and 245 transects in the reserve site
for fish and invertebrate surveys, respectively. Similar sampling effort
was applied to the control site, with 221 fish and 222 invertebrate
transects. Between 12 and 27 transects were performed in each site every
year.

Socioeconomic data were obtained from the National Commission for
Aquaculture and Fisheries (\emph{Comisión Nacional de Acuacultura y
Pesca}; CONAPESCA). The data contains \DIFdelbegin \DIFdel{species--level }\DIFdelend \DIFaddbegin \DIFadd{species-level }\DIFaddend information on
monthly landings and income from nine species from 2000 to 2014. Data on
landings and income were aggregated by year and species, and adjusted by
the Consumer Price Index {[}\DIFdelbegin \DIFdel{42}\DIFdelend \DIFaddbegin \DIFadd{48}\DIFaddend {]}. From the nine species available, we
selected as objective species those that contributed the most (88.27\%)
income from 2000 to 2014: lobster (\emph{Panulirus interruptus};
71.76\%), red sea urchin (\emph{Mesocentrotus franciscanus}; 9.33\%),
snail (\emph{Megastraea undosa}; 3.93\%), and sea cucumber
(\emph{Parastichopus parvimensis}; 3.23\%). Abalone species
(\emph{Haliotis fulgens}; 4.52\% and \emph{Haliotis corrugata}; 6.16\%)
were excluded because the cooperative implemented an informal closure of
these fisheries in 2010 to allow the population \DIFaddbegin \DIFadd{to }\DIFaddend recover. Eliminating
all fishing pressure on abalones means that the control site receives
(for this species) the same treatment as the reserve.

We constructed the governance data based on local knowledge of the area
and the community.

\section{Results from illustrative
example}\label{results-from-illustrative-example}

In this section we show the results of the application of MAREA to the
La Plana/Las Cuevas marine reserve in Isla Natividad, Mexico. These
results are intended to highlight the relevance and utility of the MAREA
framework and app, which automate the analysis and make it replicable.
While we highlight some of the general observed trends, we focus on the
utility of the tool rather than on the specific effectiveness of this
case study marine reserve.

The scorecard (Fig. 4) shows that this reserve achieves a general score
of 64\%, suggesting that 64\% of all indicators are positive. All
category--level scores were also high, with values of 67\%, 60\%, and
71\% positive indicators for biological, socioeconomic and governance,
respectively.

\textbf{Fig 4. Scorecard produced by MAREA for the ``La Plana / Las
Cuevas'' marine reserve in Isla Natividad, Mexico.}

Among the biological indicators, the greatest effect of the reserve was
observed for snail and sea cucumber densities, with values of
\(\beta_2 = 97.17\) (\emph{p} \textless{} 0.05) and \(\beta_2 = 2.31\)
(\emph{p} \textless{} 0.05), respectively. Fish indicators showed no
significant change (\emph{p} \textgreater{} 0.1), with negative trends
for Shannon's diversity index and fish species richness and positive
trends for density, biomass, and mean trophic level. Changes through
time for these indicators are presented in Figure 5, and a summary of
\(\beta_2\) coefficients is provided in Table 3.

\DIFdelbegin %DIFDELCMD < \clearpage
%DIFDELCMD < 

%DIFDELCMD < %%%
\DIFdelend \textbf{Fig. 5 Plots for values of each biological indicator (y-axis)
through time (x-axis).} Red and blue correspond to the reserve and
control sites, respectively. Black lines indicate yearly mean values,
and ribbons indicate \(\pm\) 1 standard error. Dots are horizontally
jittered to aid visualization. This figure contains information for fish
Shannon's diversity index (a), fish species richness (b), fish density
(c), fish trophic level (d), fish biomass (e), invertebrate Shannon's
diversity index (f), invertebrate species richness (g), invertebrate
density (h), lobster density (i), urchin density (j), snail density (k),
and sea cucumber density (l).

\DIFdelbegin %DIFDELCMD < \begin{table}
%DIFDELCMD < %%%
\DIFdelendFL \DIFaddbeginFL \textbf{\DIFaddFL{Table 3. Summary of average treatment effect of the reserve on
biological indicators.}}
\DIFaddendFL 

\DIFdelbeginFL %DIFDELCMD < \caption{%
{%DIFAUXCMD
%DIFDELCMD < \label{tab:table of bio results}%%%
\DIFdelFL{Summary of average treatment effect of the reserve on biological indicators.}}
%DIFAUXCMD
\DIFdelendFL \DIFaddbeginFL \begin{table}[H]
\DIFaddendFL \centering
\DIFdelbeginFL %DIFDELCMD < \begin{tabular}[t]{l|l|r}
%DIFDELCMD < %%%
\DIFdelendFL \DIFaddbeginFL \begin{tabular}{l|l|r}
\DIFaddendFL \hline
\DIFdelbeginFL \DIFdelFL{Indicator }\DIFdelendFL \DIFaddbeginFL \bfseries{Indicator} \DIFaddendFL & \DIFdelbeginFL \DIFdelFL{Estimate (SD) }\DIFdelendFL \DIFaddbeginFL \bfseries{Estimate (SD)} \DIFaddendFL & \DIFdelbeginFL \DIFdelFL{t-score}\DIFdelendFL \DIFaddbeginFL \bfseries{t-score}\DIFaddendFL \\
\hline
Shannon fish & -0.22 (0.16) & -1.3969\\
\hline
Richness fish & -0.61 (0.43) & -1.4073\\
\hline
Density fish & 0.74 (6.15) & 0.1205\\
\hline
Trophic fish & 0.00 (0.01) & 0.1399\\
\hline
Biomass fish & 0.22 (1.47) & 0.1476\\
\hline
Shannon invert & -0.67 (0.22)** & -3.0481\\
\hline
Richness invert & -2.71 (0.81)** & -3.3519\\
\hline
Density invert & 91.21 (47.11)* & 1.9362\\
\hline
Lobster & 7.66 (8.93) & 0.8583\\
\hline
Urchin & 2.15 (1.23)* & 1.7425\\
\hline
Snail & 97.17 (42.90)** & 2.2652\\
\hline
Cucumber & 2.31 (1.17)** & 1.9782\\
\hline
\DIFdelbeginFL %DIFDELCMD < \multicolumn{3}{l}{\textsuperscript{*} Asterisks indicate significance level, with (*)}\\
%DIFDELCMD < \multicolumn{3}{l}{indicating p < 0.1 and (**) p < 0.05.}\\
%DIFDELCMD < %%%
\DIFdelendFL \end{tabular}
\end{table}

\DIFaddbegin \DIFadd{* Indicate significance level, with (*) indicating p \textless{} 0.1 and
(**) p \textless{} 0.05.
}

\clearpage

\DIFaddend One of the main objectives of this reserve was to increase landings.
Results of the socioeconomic indicators show that total landings were,
on average, 64.20 metric tonnes higher (\emph{p} \textgreater{} 0.1)
after the implementation of the reserves, though this cannot necessarily
be interpreted as causal, because it relies entirely on a \DIFdelbegin \DIFdel{before--after
}\DIFdelend \DIFaddbegin \DIFadd{before-after
}\DIFaddend comparison. Total income was \$10,344.85 (\emph{p} \textless{} 0.05)
thousands of Mexican Pesos (K MXP) higher after the implementation of
the reserves. On average, lobster and sea cucumber landings increased,
while urchin and snail landings and income decreased. Figure 6 presents
the changes in these indicators through time, and Table 4 summarizes
these results.

\textbf{Fig. 6 Plots for values of each socioeconomic indicator (y-axis)
through time (x-axis).} Red and blue correspond to before and after the
implementation of the reserve, respectively. This figure contains
information for total landings (a), total income (b), lobster landings
(c), urchin landings (d), snail landings (e), sea cucumber landings (f),
lobster income (g), urchin income (h), snail income (i), and sea
cucumber income (j).

\DIFdelbegin %DIFDELCMD < \begin{table}
%DIFDELCMD < %%%
\DIFdelendFL \DIFaddbeginFL \textbf{\DIFaddFL{Table 4. Summary of differences in socioeconomic indicators
before and after the implementation of the reserve.}}
\DIFaddendFL 

\DIFdelbeginFL %DIFDELCMD < \caption{%
{%DIFAUXCMD
%DIFDELCMD < \label{tab:table of soc results}%%%
\DIFdelFL{Summary of differences in socioeconomic indicators before and after the implementation of the reserve.}}
%DIFAUXCMD
\DIFdelendFL \DIFaddbeginFL \begin{table}[H]
\DIFaddendFL \centering
\DIFdelbeginFL %DIFDELCMD < \begin{tabular}[t]{l|l|r}
%DIFDELCMD < %%%
\DIFdelendFL \DIFaddbeginFL \begin{tabular}{l|l|r}
\DIFaddendFL \hline
\DIFdelbeginFL \DIFdelFL{Indicator }\DIFdelendFL \DIFaddbeginFL \bfseries{Indicator} \DIFaddendFL & \DIFdelbeginFL \DIFdelFL{Estimate (SD) }\DIFdelendFL \DIFaddbeginFL \bfseries{Estimate (SD)} \DIFaddendFL & \DIFdelbeginFL \DIFdelFL{t-score}\DIFdelendFL \DIFaddbeginFL \bfseries{t-score}\DIFaddendFL \\
\hline
Landings & 64.20 (90.07) & 0.7127\\
\hline
Income & 10344.85 (3982.20)** & 2.5978\\
\hline
Lobster landings & 7.37 (13.95) & 0.5281\\
\hline
Urchin landings & -30.00 (9.49)** & -3.1620\\
\hline
Snail landings & -69.53 (33.82)* & -2.0561\\
\hline
Cucumber landings & 9.34 (6.72) & 1.3906\\
\hline
Lobster income & 14372.85 (3634.64)** & 3.9544\\
\hline
Urchin income & -5800.46 (1867.50)** & -3.1060\\
\hline
Snail income & -404.85 (187.07)** & -2.1641\\
\hline
Cucumber income & 131.49 (185.66) & 0.7082\\
\hline
\DIFdelbeginFL %DIFDELCMD < \multicolumn{3}{l}{\textsuperscript{*} Asterisks indicate significance level, with (*)}\\
%DIFDELCMD < \multicolumn{3}{l}{indicating p < 0.1 and (**) p < 0.05.}\\
%DIFDELCMD < %%%
\DIFdelendFL \end{tabular}
\end{table}

\DIFaddbegin \DIFadd{* Indicate significance level, with (*) indicating p \textless{} 0.1 and
(**) p \textless{} 0.05.
}

\DIFaddend Recall that the governance objectives are evaluated based on the
institutions present, not on a specific quantitative linkage between
governance and biological or economic outcomes. Data for this reserve
suggest that the community is strongly organized, which is a likely
driver of the successes reported above {[}\DIFdelbegin \DIFdel{55}\DIFdelend \DIFaddbegin \DIFadd{62}\DIFaddend {]}. The first point of
success is the existence of a fishing cooperative that is also
affiliated with a regional federation of cooperatives. These polycentric
governance structures allow various levels of organization that have
been shown to foster communication and cooperation {[}\DIFdelbegin \DIFdel{43,}\DIFdelend 50\DIFaddbegin \DIFadd{,57}\DIFaddend {]};
federations also provide bargain power with governments {[}\DIFdelbegin \DIFdel{43,56}\DIFdelend \DIFaddbegin \DIFadd{50,63}\DIFaddend {]}.
Access to fishing resources is managed through a TURF, permits, and
fishing quotas (for some species). McCay {[}\DIFdelbegin \DIFdel{49}\DIFdelend \DIFaddbegin \DIFadd{56}\DIFaddend {]} suggests that the TURF
promotes a sense of stewardship of their resources and incentivizes
sustainable management. Together, these structures enabled a
participative, bottom--up process during the reserve design phase;
opinions of all fishing members \DIFdelbegin \DIFdel{---and often non--fishing community
members--- }\DIFdelend \DIFaddbegin \DIFadd{--and often non-fishing community
members-- }\DIFaddend were included. Participation of community members in reserve
surveillance and yearly monitoring indicate commitment and interest, and
allow informal communication of results to \DIFdelbegin \DIFdel{un--involved }\DIFdelend \DIFaddbegin \DIFadd{uninvolved }\DIFaddend community members.
Furthermore, the reserve is partially isolated from poaching activity\DIFaddbegin \DIFadd{,
}\DIFaddend and fishers have internal regulations pertaining to the reserves. The
low level of illegal fishing by members of the community and outsiders
both inside and outside the reserve represents another indication of
effectiveness. Governance indicators are summarized in Table 5.

\DIFdelbegin %DIFDELCMD < \begin{table}
%DIFDELCMD < %%%
\DIFdelendFL \DIFaddbeginFL \clearpage
\DIFaddendFL 

\DIFdelbeginFL %DIFDELCMD < \caption{%
{%DIFAUXCMD
%DIFDELCMD < \label{tab:unnamed-chunk-6}%%%
\DIFdelFL{Summary of governance indicators.}}
%DIFAUXCMD
\DIFdelendFL \DIFaddbeginFL \textbf{\DIFaddFL{Table 5. Summary of governance indicators.}}

\begin{table}[H]
\DIFaddendFL \centering
\DIFdelbeginFL %DIFDELCMD < \begin{tabular}[t]{l|>{\raggedright\arraybackslash}p{9cm}}
%DIFDELCMD < %%%
\DIFdelendFL \DIFaddbeginFL \begin{tabular}{l|>{\raggedright\arraybackslash}p{9cm}}
\DIFaddendFL \hline
\DIFdelbeginFL \DIFdelFL{Indicator }\DIFdelendFL \DIFaddbeginFL \bfseries{Indicator} \DIFaddendFL & \DIFdelbeginFL \DIFdelFL{Description}\DIFdelendFL \DIFaddbeginFL \bfseries{Description}\DIFaddendFL \\
\hline
Access to the fishery & Permits, Territorial Use Rights for Fisheries, Quotas (for some fisheries)\\
\hline
Number of fishers & Stable\\
\hline
Legal recognition of reserve & Not recognized\\
\hline
Reserve type & Community-based Marine Reserve\\
\hline
Illegal harvesting & Due to its relative isolations, neither the reserve or TURF suffer from significant illegal harvesting\\
\hline
Management plan & The reserve does not have a management plan, but written rules exist within the cooperative\\
\hline
Reserve enforcement & Fishers have two land stations equipped with radars and patrol boats 24/7 to patrol the reserves.\\
\hline
Size of reserve & The reserve is big enough to protect the targeted sessile or not highly mobile invertebrates (lobster, urchin, snail, cucumber, and abalone)\\
\hline
Reasoning for reserve location & The reserves were put in place in zones that, according to local knowledge, were once very productive. Habitat heterogeneity and ease of monitoring, surveillance and enforcement were also considered.\\
\hline
Membership to fisher organizations & The fishers are part of fisher organizations.\\
\hline
Type of fisheries organizations & The fishers are part of a cooperative (S.C.P.P. Buzos y Pescadores de la Baja California) and are affiliated to a federation (FEDECOOP).\\
\hline
Representation & Reserves were designed by fishers in a bottom-up approach, incorporating expertise from academics and CSO members. This was a highly inclusive and participatory process.\\
\hline
Internal Regulation & Fishers have stringent internal regulations to control fishing effort throughout their TURF, assigning different fishing zones and gears to different teams. Rules pertaining the marine reserves also exist.\\
\hline
Perceived Effectiveness & The fishers have a positive perception about the effectiveness of their reserve, often stating that they have seen significant economic benefits.\\
\hline
Social Impact of Reserve & The reserves have had a significant positive social impact. Fishers are proud to be an example of successgul marine conservation, allowing them to have increased social capital.\\
\hline
\end{tabular}
\end{table}

\clearpage

\section{Discussion}\label{discussion}

We have developed and presented \DIFdelbegin \DIFdel{a user--friendly, }\DIFdelend \DIFaddbegin \DIFadd{an }\DIFaddend automated approach for evaluating the
effectiveness of marine reserves \DIFaddbegin \DIFadd{in Mexico, and perhaps }\DIFaddend around the
world. Here we highlight MAREA's utility for \DIFdelbegin \DIFdel{evidence---based }\DIFdelend \DIFaddbegin \DIFadd{evidence-based }\DIFaddend management,
and comment on a few of its shortcomings. The findings from Isla
Natividad are used purely to validate the relevance of MAREA rather than
to discuss particularities of the marine reserve effectiveness, which
has been described before {[}\DIFdelbegin \DIFdel{9,11,53}\DIFdelend \DIFaddbegin \DIFadd{8,10,43}\DIFaddend {]}. We use examples from the case
study to build on the utility of MAREA and discuss ways in which results
can be interpreted to inform management.

The causal inference techniques used by MAREA have been suggested
{[}\DIFdelbegin \DIFdel{37,41}\DIFdelend \DIFaddbegin \DIFadd{38,47}\DIFaddend {]} and used {[}\DIFdelbegin \DIFdel{20}\DIFdelend \DIFaddbegin \DIFadd{21}\DIFaddend {]} before in other ad hoc studies. This
approach reduces ambiguity in the interpretation of results. For
example, invertebrate density decreased through time inside and outside
of the reserve (Fig. 5h). In this case, a before--after evaluation of
the reserve (\emph{i.e.} ignoring the control site) would have
incorrectly concluded that the reserve failed to protect invertebrates.
On the other hand, a control--impact approach (\emph{i.e.} compare
reserve vs.~control site only in 2016) would have identified higher
densities inside the reserve, concluding that the reserve increases
invertebrate density. However, by executing a formal
\DIFdelbegin \DIFdel{difference--in--differences }\DIFdelend \DIFaddbegin \DIFadd{difference-in-differences }\DIFaddend approach for causal inference, MAREA
identifies the changes through time and across sites, and estimates the
effect of the reserve on density at \(\beta_2 = 91.21\) (\emph{p}
\textless{} 0.05). This approach reveals that invertebrate densities
decrease in both sites through time, but the decrease is faster for the
control site, thus yielding a positive value for \(\beta_2\).

The approach used by MAREA to estimate the effect of the reserve on
biological indicators requires cautious interpretation of the results.
The value of the \(\beta_2\) coefficient represents the difference
between the temporal trends of the reserve and control sites {[}\DIFdelbegin \DIFdel{20}\DIFdelend \DIFaddbegin \DIFadd{21}\DIFaddend {]}.
As exemplified by the case of invertebrate densities, a positive value
(\emph{i.e.} \(\beta_2 > 0\)) does not necessarily indicate an increase
in the indicator through time, but rather a positive difference with
respect to the temporal trend of the control site. The inverse occurs
for negative values of \(\beta_2\).

MAREA provides \DIFdelbegin \DIFdel{in--depth }\DIFdelend \DIFaddbegin \DIFadd{in-depth }\DIFaddend analysis and a convenient snapshot overview of
the effect of the reserve, allowing users to rapidly identify trends.
However, users must interpret multiple indicators at a time to better
understand the results. For example, with additional knowledge of local
environmental variability (\emph{i.e.} indicator B5: Natural
Disturbance), we can better understand the trends in invertebrate
densities. As reported before {[}\DIFdelbegin \DIFdel{9}\DIFdelend \DIFaddbegin \DIFadd{8}\DIFaddend {]}, hypoxic conditions that have
occurred in Isla Natividad can cause decreases in invertebrate
densities, and reserves buffer the negative effect. While MAREA
automates the analysis and makes results replicable, proper
interpretation will still depend on the user. Results produced by MAREA
can only aid in management and decision making when results have been
correctly interpreted.

Socioeconomic and governance indicators typically lack a control site,
which impede us from using the causal inference techniques employed to
measure biological changes {[}\DIFdelbegin \DIFdel{24}\DIFdelend \DIFaddbegin \DIFadd{25}\DIFaddend {]}. However, we can still extract
useful information from them. Again, by combining results from multiple
indicators, MAREA can provide insights into the effect of the reserve.
For example, lobster and sea cucumber have shown increases in densities,
landings, and income. We cannot conclude that landings and income from
these species have increased due to the reserve, but we can at least
conclude that landings have not decreased. While further information on
market behavior of each fishery is needed, these results provide
insights into the state of the reserve and its associated fisheries.

As for the governance information, it is difficult to establish causal
links between the state of the reserve and the governance structures
present in the community. However, providing a single platform
(\emph{i.e.} scorecard) or document (\emph{i.e.} technical report) where
biological, socioeconomic, and governance information is comprehensively
included can aid in management. By using MAREA, this information will be
reported across reserves in a standardized way, and can help managers
identify overarching patterns across sites.

By making results straightforward to interpret, MAREA may also assist in
communication with a broader stakeholder community. While stakeholder
involvement in the design and implementation phases of marine reserves
is important, that may not be sufficient for ensuring \DIFdelbegin \DIFdel{long--run buy--in
}\DIFdelend \DIFaddbegin \DIFadd{long-term buy-in
}\DIFaddend or success. The scorecard is easily understandable by experts and
\DIFdelbegin \DIFdel{non--experts}\DIFdelend \DIFaddbegin \DIFadd{non-experts}\DIFaddend , and can be used as an effective tool for communicating the
results of annual evaluations. Additionally, the technical report can
serve as a tool for managers and scientists to rapidly produce and
communicate information at a more technical level.

We recognize that the \DIFaddbegin \DIFadd{seven objectives and }\DIFaddend 29 indicators used by MAREA
might not fully describe a reserve \DIFdelbegin \DIFdel{. However, they }\DIFdelend \DIFaddbegin \DIFadd{in countries other than Mexico. In
order to ensure the applicability of the tool to reserves in other
countries, further testing in other regions should take place. However,
the proposed objectives and indicators }\DIFaddend provide a starting point to
perform the evaluation, to which managers and users can add other
indicators (\emph{e.g.} larval dispersal or connectivity) that are
relevant to their reserve. Furthermore, MAREA's value is that it
provides a free, simple, and replicable way to perform rigorous impact
analysis. The tool can easily be used by fishers, CSO members, and
managers in government agencies, providing transparency of the analysis
and results. In addition, it can empower and enable local managers and
fishers to respond to local change and adapt by allowing direct and easy
access to the information.

\DIFaddbegin \DIFadd{An evident limitation of MAREA is its dependence on data obtained
through a BACI design, and the amount of samples needed to estimate
coefficients in Eqn. 1. It is not uncommon for control sites or
baselines to be absent. Properly designing marine reserves by
identifying control sites and establishing a baseline before the
implementation of the reserve is enough to overcome this issue; reserves
for which there is no control site and baseline cannot be evaluated with
MAREA. Typical underwater surveys require that at least 12 - 16
transects are performed for each site (}\emph{\DIFadd{i.e.}} \DIFadd{reserve and control)
each year. This provides at least 48 samples (12 samples per site, per
year), enough to avoid overfitting Eqn 1. However, these problems can be
easily avoided during the design and implementation phases by
anticipating what data will be needed in the eventual evaluation.
}

\DIFadd{To the best of our knowledge, MAREA is the first tool designed to
evaluate marine reserves. Previous work }{[}\DIFadd{23,25}{]} \DIFadd{addressed MPA
evaluation and provided the foundation for our contribution. However,
these did not intended to create user-friendly tools to aid in the
evaluation. As with other conservation management tools, development of
tools that automatize complex calculations can have an important impact
in management }{[}\DIFadd{64}{]}\DIFadd{. The use of open data science enables the
creation of open-access tools that can address technical gaps and
inprove management }{[}\DIFadd{28}{]}\DIFadd{.
}

\DIFaddend The effectiveness of marine reserves continues to be a matter of debate
{[}\DIFdelbegin \DIFdel{7,12,40}\DIFdelend \DIFaddbegin \DIFadd{11,44,65}\DIFaddend {]}. With current targets set to increase ocean protection,
it is important that we understand the effects of our interventions
{[}\DIFdelbegin \DIFdel{37}\DIFdelend \DIFaddbegin \DIFadd{38}\DIFaddend {]} so we can better inform management {[}\DIFdelbegin \DIFdel{41}\DIFdelend \DIFaddbegin \DIFadd{47}\DIFaddend {]}. It is therefore
important that academics, managers, fishers, and CSOs have access to
open access tools like MAREA. This is particularly relevant for Mexico
and other Latin American countries, where management agencies are often
understaffed and underfunded {[}\DIFdelbegin \DIFdel{57}\DIFdelend \DIFaddbegin \DIFadd{66}\DIFaddend {]}, or where materials are often not
available in their language. In this context, MAREA provides a simple
and replicable way to align management objectives with performance
indicators. The proposed methodologies, especially the way in which
biological indicators are evaluated, provide valuable information for
managers. We acknowledge there is room for improvement in the way in
which socioeconomic and governance data are analyzed. Despite this,
providing a unifying platform where all indicators can be analyzed and
comprehensively presented represents a valuable step towards effective
evidence---based management {[}\DIFdelbegin \DIFdel{41}\DIFdelend \DIFaddbegin \DIFadd{47}\DIFaddend {]}.

The first release of MAREA is now available, and it will continue to be
developed and maintained to keep up to date with the literature. This
process will incorporate new features, and enhance current ones, aiming
to improve user experience and expand the scope of the analysis. \DIFaddbegin \DIFadd{Other
modifications may also include addition of more objectives and
indicators to ensure applicability in other regions, full translation
into other languages to avoid any ambiguities introduced via the
automatic translation, or reporting effects over time in percentages to
aid interpretation. }\DIFaddend Yet, we believe that this first release represents a
major step towards effective, replicable evaluation and management of
marine reserves.

\section{Acknowledgements}\label{acknowledgements}

We thank Olivier Deschenes and Andrew Plantinga, who provided valuable
input to design the model that evaluates the biological indicators.
Special thanks to the fishers from Isla Natividad, who gathered the data
used in this study, and the fishers from El Rosario, who helped us
validate our survey and framework, and to Arturo Hernández and Alfonso
Romero who provided help with the logistics. \DIFaddbegin \DIFadd{Finally, we thank the
editor and two anonymous reviewers for their suggestions, which
significantly improved the quality of this paper.
}\DIFaddend 

\clearpage

\section{References}\label{references}

\hypertarget{refs}{}
\hypertarget{ref-pauly_2005-qV}{}
1. Pauly D, Watson R, Alder J. Global trends in world fisheries: Impacts
on marine ecosystems and food security. \DIFdelbegin \DIFdel{Philos Trans R Soc Lond, B, Biol
Sci}\DIFdelend \DIFaddbegin \DIFadd{Philosophical Transactions of
the Royal Society B: Biological Sciences}\DIFaddend . 2005;360: 5--12.
doi:\href{https://doi.org/10.1098/rstb.2004.1574}{10.1098/rstb.2004.1574}

\hypertarget{ref-halpern_2008-dK}{}
2. Halpern BS, Walbridge S, Selkoe KA, Kappel CV, Micheli F, D'Agrosa C,
et al. A global map of human impact on marine ecosystems. Science.
2008;319: 948--952.
doi:\href{https://doi.org/10.1126/science.1149345}{10.1126/science.1149345}

\hypertarget{ref-lester_2008-F_}{}
3. Lester S, Halpern B. Biological responses in marine no-take reserves
versus partially protected areas. Mar Ecol Prog Ser. 2008;367: 49--56.
doi:\href{https://doi.org/10.3354/meps07599}{10.3354/meps07599}

\hypertarget{ref-lester_2009-Ks}{}
4. Lester S, Halpern B, Grorud-Colvert K, Lubchenco J, Ruttenberg B,
Gaines S, et al. Biological effects within no-take marine reserves: A
global synthesis. Mar Ecol Prog Ser. 2009;384: 33--46.
doi:\href{https://doi.org/10.3354/meps08029}{10.3354/meps08029}

\hypertarget{ref-sala_2016-PV}{}
5. Sala E, Costello C, De Bourbon Parme J, Fiorese M, Heal G, Kelleher
K, et al. Fish banks: An economic model to scale marine conservation.
Marine Policy. 2016;73: 154--161.
doi:\href{https://doi.org/10.1016/j.marpol.2016.07.032}{10.1016/j.marpol.2016.07.032}

\hypertarget{ref-hastings_2017-sm}{}
6. Hastings A, Gaines SD, Costello C. Marine reserves solve an important
bycatch problem in fisheries. Proc Natl Acad Sci \DIFdelbegin \DIFdel{U S A}\DIFdelend \DIFaddbegin \DIFadd{USA}\DIFaddend . 2017;\DIFdelbegin \DIFdel{doi:
}\DIFdelend \DIFaddbegin \DIFadd{114:
8927--8934.
doi:}\DIFaddend \href{https://doi.org/10.1073/pnas.1705169114}{10.1073/pnas.1705169114}

\DIFdelbegin %DIFDELCMD < \hypertarget{ref-padleton_2017-vn}{}
%DIFDELCMD < %%%
\DIFdel{7. Padleton L, Aghmadia G, Browman H, Thurstand R, Kaplan D, Bartolino
V.
Debating the effectiveness of marine protected areas. ICES Journal of
Marine Science. 2017;
doi:}%DIFDELCMD < \href{https://doi.org/10.1093/icesjms/fsx154}{10.1093/icesjms/fsx154}
%DIFDELCMD < 

%DIFDELCMD < %%%
\DIFdelend \hypertarget{ref-aburtooropeza_2011-ya}{}
\DIFdelbegin \DIFdel{8. }\DIFdelend \DIFaddbegin \DIFadd{7. }\DIFaddend Aburto-Oropeza O, Erisman B, Galland GR, Mascareñas-Osorio I, Sala E,
Ezcurra E. Large recovery of fish biomass in a no-take marine reserve.
PLoS ONE. 2011;6: e23601.
doi:\href{https://doi.org/10.1371/journal.pone.0023601}{10.1371/journal.pone.0023601}

\hypertarget{ref-micheli_2012-EU}{}
\DIFdelbegin \DIFdel{9. }\DIFdelend \DIFaddbegin \DIFadd{8. }\DIFaddend Micheli F, Saenz-Arroyo A, Greenley A, Vazquez L, Espinoza Montes JA,
Rossetto M, et al. Evidence that marine reserves enhance resilience to
climatic impacts. PLoS ONE. 2012;7: e40832.
doi:\href{https://doi.org/10.1371/journal.pone.0040832}{10.1371/journal.pone.0040832}

\hypertarget{ref-roberts_2017-J9}{}
\DIFdelbegin \DIFdel{10. }\DIFdelend \DIFaddbegin \DIFadd{9. }\DIFaddend Roberts CM, \DIFdelbegin \DIFdel{O'Leary }\DIFdelend \DIFaddbegin \DIFadd{OLeary }\DIFaddend BC, McCauley DJ, Cury PM, Duarte CM, Lubchenco J,
et al. Marine reserves can mitigate and promote adaptation to climate
change. Proc Natl Acad Sci \DIFdelbegin \DIFdel{U S A}\DIFdelend \DIFaddbegin \DIFadd{USA}\DIFaddend . 2017;114: 6167--6175.
doi:\href{https://doi.org/10.1073/pnas.1701262114}{10.1073/pnas.1701262114}

\hypertarget{ref-munguavega_2015-yg}{}
\DIFdelbegin \DIFdel{11. }\DIFdelend \DIFaddbegin \DIFadd{10. }\DIFaddend Munguía-Vega A, Sáenz-Arroyo A, Greenley AP, Espinoza-Montes JA,
Palumbi SR, Rossetto M, et al. Marine reserves help preserve genetic
diversity after impacts derived from climate variability: Lessons from
the pink abalone in baja california. Global Ecology and Conservation.
2015;4: 264--276.
doi:\href{https://doi.org/10.1016/j.gecco.2015.07.005}{10.1016/j.gecco.2015.07.005}

\hypertarget{ref-edgar_2014-UO}{}
\DIFdelbegin \DIFdel{12. }\DIFdelend \DIFaddbegin \DIFadd{11. }\DIFaddend Edgar GJ, Stuart-Smith RD, Willis TJ, Kininmonth S, Baker SC, Banks
S, et al. Global conservation outcomes depend on marine protected areas
with five key features. Nature. 2014;506: 216--220.
doi:\href{https://doi.org/10.1038/nature13022}{10.1038/nature13022}

\hypertarget{ref-giakoumi_2017-V2}{}
\DIFdelbegin \DIFdel{13. }\DIFdelend \DIFaddbegin \DIFadd{12. }\DIFaddend Giakoumi S, Scianna C, Plass-Johnson J, Micheli F, Grorud-Colvert K,
Thiriet P, et al. Ecological effects of full and partial protection in
the crowded mediterranean sea: A regional meta-analysis. Sci Rep.
2017;7: 8940.
doi:\href{https://doi.org/10.1038/s41598-017-08850-w}{10.1038/s41598-017-08850-w}

\hypertarget{ref-sala_2017-69}{}
\DIFdelbegin \DIFdel{14. }\DIFdelend \DIFaddbegin \DIFadd{13. }\DIFaddend Sala E, Giakoumi S. No-take marine reserves are the most effective
protected areas in the ocean. ICES Journal of Marine Science. 2017;
doi:\href{https://doi.org/10.1093/icesjms/fsx059}{10.1093/icesjms/fsx059}

\hypertarget{ref-betti_2017-lq}{}
\DIFdelbegin \DIFdel{15. }\DIFdelend \DIFaddbegin \DIFadd{14. }\DIFaddend Betti F, Bavestrello G, Bo M, Asnaghi V, Chiantore M, Bava S, et al.
Over 10\DIFaddbegin \DIFadd{~}\DIFaddend years of variation in mediterranean reef benthic communities.
Marine Ecology. 2017;38: e12439.
doi:\href{https://doi.org/10.1111/maec.12439}{10.1111/maec.12439}

\DIFdelbegin %DIFDELCMD < \hypertarget{ref-davies_2017-ml}{}
%DIFDELCMD < %%%
\DIFdelend \DIFaddbegin \hypertarget{ref-szuwalski_2017-jc}{}
\DIFadd{15. Szuwalski CS, Burgess MG, Costello C, Gaines SD. High fishery
catches through trophic cascades in china. Proc Natl Acad Sci USA.
2017;114: 717--721.
doi:}\href{https://doi.org/10.1073/pnas.1612722114}{10.1073/pnas.1612722114}

\hypertarget{ref-chavez_2003-_m}{}
\DIFaddend 16. \DIFaddbegin \DIFadd{Chavez FP. From anchovies to sardines and back: Multidecadal change
in the pacific ocean. Science. 2003;299: 217--221.
doi:}\href{https://doi.org/10.1126/science.1075880}{10.1126/science.1075880}

\hypertarget{ref-davies_2017-ml}{}
\DIFadd{17. }\DIFaddend Davies TK, Mees CC, Milner-Gulland EJ. Use of a counterfactual
approach to evaluate the effect of area closures on fishing location in
a tropical tuna fishery. PLoS ONE. 2017;12: e0174758.
doi:\href{https://doi.org/10.1371/journal.pone.0174758}{10.1371/journal.pone.0174758}

\hypertarget{ref-guidetti_2014-8Z}{}
\DIFdelbegin \DIFdel{17. }\DIFdelend \DIFaddbegin \DIFadd{18. }\DIFaddend Guidetti P, Baiata P, Ballesteros E, Di Franco A, Hereu B,
Macpherson E, et al. Large-scale assessment of mediterranean marine
protected areas effects on fish assemblages. PLoS ONE. 2014;9: e91841.
doi:\href{https://doi.org/10.1371/journal.pone.0091841}{10.1371/journal.pone.0091841}

\hypertarget{ref-friedlander_2017-oI}{}
\DIFdelbegin \DIFdel{18. }\DIFdelend \DIFaddbegin \DIFadd{19. }\DIFaddend Friedlander AM, Golbuu Y, Ballesteros E, Caselle JE, Gouezo M,
Olsudong D, et al. Size, age, and habitat determine effectiveness of
palau's marine protected areas. PLoS ONE. 2017;12: e0174787.
doi:\href{https://doi.org/10.1371/journal.pone.0174787}{10.1371/journal.pone.0174787}

\hypertarget{ref-rodriguez_2017-PD}{}
\DIFdelbegin \DIFdel{19. }\DIFdelend \DIFaddbegin \DIFadd{20. }\DIFaddend Rodriguez AG, Fanning LM. Assessing marine protected areas
effectiveness: A case study with the tobago cays marine park. OJMS.
2017;07: 379--408.
doi:\href{https://doi.org/10.4236/ojms.2017.73027}{10.4236/ojms.2017.73027}

\hypertarget{ref-moland_2013-VP}{}
\DIFdelbegin \DIFdel{20. }\DIFdelend \DIFaddbegin \DIFadd{21. }\DIFaddend Moland E, Olsen EM, Knutsen H, Garrigou P, Espeland SH, Kleiven AR,
et al. Lobster and cod benefit from small-scale northern marine
protected areas: Inference from an empirical before-after control-impact
study. \DIFdelbegin \DIFdel{Proc Biol Sci}\DIFdelend \DIFaddbegin \DIFadd{Proceedings of the Royal Society B: Biological Sciences}\DIFaddend .
2013;280: \DIFdelbegin \DIFdel{20122679.
}\DIFdelend \DIFaddbegin \DIFadd{20122679--20122679.
}\DIFaddend doi:\href{https://doi.org/10.1098/rspb.2012.2679}{10.1098/rspb.2012.2679}

\hypertarget{ref-soykan_2015-nu}{}
\DIFdelbegin \DIFdel{21. }\DIFdelend \DIFaddbegin \DIFadd{22. }\DIFaddend Soykan CU, Lewison RL. Using community-level metrics to monitor the
effects of marine protected areas on biodiversity. Conserv Biol.
2015;29: 775--783.
doi:\href{https://doi.org/10.1111/cobi.12445}{10.1111/cobi.12445}

\hypertarget{ref-pomeroy_2005-Py}{}
\DIFdelbegin \DIFdel{22. }\DIFdelend \DIFaddbegin \DIFadd{23. }\DIFaddend Pomeroy RS, Watson LM, Parks JE, Cid GA. How is your mpa doing? A
methodology for evaluating the management effectiveness of marine
protected areas. Ocean Coast Manag. 2005;48: 485--502.
doi:\href{https://doi.org/10.1016/j.ocecoaman.2005.05.004}{10.1016/j.ocecoaman.2005.05.004}

\hypertarget{ref-pomeroy_2004-23}{}
\DIFdelbegin \DIFdel{23. }\DIFdelend \DIFaddbegin \DIFadd{24. }\DIFaddend Pomeroy RS, Parks JE, Watson LM. How is your mpa doing ? A guidebook
of natural and social indicators for evaluating marine protected areas
management effectiveness {[}Internet{]}. IUCN; 2004.
doi:\href{https://doi.org/10.2305/IUCN.CH.2004.PAPS.1.en}{10.2305/IUCN.CH.2004.PAPS.1.en}

\hypertarget{ref-mascia_2017-m_}{}
\DIFdelbegin \DIFdel{24. }\DIFdelend \DIFaddbegin \DIFadd{25. }\DIFaddend Mascia MB, Fox HE, Glew L, Ahmadia GN, Agrawal A, Barnes M, et al. A
novel framework for analyzing conservation impacts: Evaluation, theory,
and marine protected areas. Ann N Y Acad Sci. 2017;1399: 93--115.
doi:\href{https://doi.org/10.1111/nyas.13428}{10.1111/nyas.13428}

\hypertarget{ref-ostrom_2009-hg}{}
\DIFdelbegin \DIFdel{25. }\DIFdelend \DIFaddbegin \DIFadd{26. }\DIFaddend Ostrom E. A general framework for analyzing sustainability of
social-ecological systems. Science. 2009;325: 419--422.
doi:\href{https://doi.org/10.1126/science.1172133}{10.1126/science.1172133}

\hypertarget{ref-basurto_2013-oq}{}
\DIFdelbegin \DIFdel{26. }\DIFdelend \DIFaddbegin \DIFadd{27. }\DIFaddend Basurto X, Gelcich S, Ostrom E. The social--ecological system
framework as a knowledge classificatory system for benthic small-scale
fisheries. Global Environmental Change. 2013;23: 1366--1380.
doi:\href{https://doi.org/10.1016/j.gloenvcha.2013.08.001}{10.1016/j.gloenvcha.2013.08.001}

\hypertarget{ref-lowndes_2017-xh}{}
\DIFdelbegin \DIFdel{27. }\DIFdelend \DIFaddbegin \DIFadd{28. }\DIFaddend Lowndes JSS, Best BD, Scarborough C, Afflerbach JC, Frazier MR,
\DIFdelbegin \DIFdel{O'Hara }\DIFdelend \DIFaddbegin \DIFadd{OHara }\DIFaddend CC, et al. Our path to better science in less time using open data
science tools. Nat ecol evol. 2017;1: 0160.
doi:\href{https://doi.org/10.1038/s41559-017-0160}{10.1038/s41559-017-0160}

\hypertarget{ref-halpern_2012-k9}{}
\DIFdelbegin \DIFdel{28. }\DIFdelend \DIFaddbegin \DIFadd{29. }\DIFaddend Halpern BS, Longo C, Hardy D, McLeod KL, Samhouri JF, Katona SK, et
al. An index to assess the health and benefits of the global ocean.
Nature. 2012;488: 615--620.
doi:\href{https://doi.org/10.1038/nature11397}{10.1038/nature11397}

\hypertarget{ref-halpern_2017-Zi}{}
\DIFdelbegin \DIFdel{29. }\DIFdelend \DIFaddbegin \DIFadd{30. }\DIFaddend Halpern BS, Frazier M, Afflerbach J, \DIFdelbegin \DIFdel{O'Hara }\DIFdelend \DIFaddbegin \DIFadd{OHara }\DIFaddend C, Katona S, Stewart
Lowndes JS, et al. Drivers and implications of change in global ocean
health over the past five years. PLoS ONE. 2017;12: e0178267.
doi:\href{https://doi.org/10.1371/journal.pone.0178267}{10.1371/journal.pone.0178267}

\hypertarget{ref-selig_2015-F9}{}
\DIFdelbegin \DIFdel{30. }\DIFdelend \DIFaddbegin \DIFadd{31. }\DIFaddend Selig ER, Frazier M, O'Leary JK, Jupiter SD, Halpern BS, Longo C, et
al. Measuring indicators of ocean health for an island nation: The ocean
health index for fiji. Ecosystem Services. 2015;16: 403--412.
doi:\href{https://doi.org/10.1016/j.ecoser.2014.11.007}{10.1016/j.ecoser.2014.11.007}

\hypertarget{ref-halpern_2014-lQ}{}
\DIFdelbegin \DIFdel{31. }\DIFdelend \DIFaddbegin \DIFadd{32. }\DIFaddend Halpern BS, Longo C, Scarborough C, Hardy D, Best BD, Doney SC, et
al. Assessing the health of the u.S. west coast with a regional-scale
application of the ocean health index. PLoS ONE. 2014;9: e98995.
doi:\href{https://doi.org/10.1371/journal.pone.0098995}{10.1371/journal.pone.0098995}

\hypertarget{ref-elfes_2014-RC}{}
\DIFdelbegin \DIFdel{32. }\DIFdelend \DIFaddbegin \DIFadd{33. }\DIFaddend Elfes CT, Longo C, Halpern BS, Hardy D, Scarborough C, Best BD, et
al. A regional-scale ocean health index for brazil. PLoS ONE. 2014;9:
e92589.
doi:\href{https://doi.org/10.1371/journal.pone.0092589}{10.1371/journal.pone.0092589}

\hypertarget{ref-anderson_2015-ND}{}
\DIFdelbegin \DIFdel{33. }\DIFdelend \DIFaddbegin \DIFadd{34. }\DIFaddend Anderson JL, Anderson CM, Chu J, Meredith J, Asche F, Sylvia G, et
al. The fishery performance indicators: A management tool for triple
bottom line outcomes. PLoS ONE. 2015;10: e0122809.
doi:\href{https://doi.org/10.1371/journal.pone.0122809}{10.1371/journal.pone.0122809}

\hypertarget{ref-dowling_2016-pO}{}
\DIFdelbegin \DIFdel{34. }\DIFdelend \DIFaddbegin \DIFadd{35. }\DIFaddend Dowling N, Wilson J, Rudd M, Babcock E, Caillaux M, Cope J, et al.
FishPath: A decision support system for assessing and managing data- and
capacity- limited fisheries. In: Quinn II T, Armstrong J, Baker M,
Heifetz J, Witherell D, editors. Assessing and managing data-limited
fish stocks. Alaska Sea Grant, University of Alaska Fairbansk; 2016.
doi:\href{https://doi.org/10.4027/amdlfs.2016.03}{10.4027/amdlfs.2016.03}

\hypertarget{ref-oyanedel_2017-TO}{}
\DIFdelbegin \DIFdel{35. }\DIFdelend \DIFaddbegin \DIFadd{36. }\DIFaddend Oyanedel R, Macy Humberstone J, Shattenkirk K, Rodriguez Van-Dyck S,
Joye Moyer K, Poon S, et al. A decision support tool for designing
turf-reserves. BMS. 2017;93: 155--172.
doi:\href{https://doi.org/10.5343/bms.2015.1095}{10.5343/bms.2015.1095}

\hypertarget{ref-vilela_2017-Zo}{}
\DIFdelbegin \DIFdel{36. }\DIFdelend \DIFaddbegin \DIFadd{37. }\DIFaddend Vilela T, Reid J. Improving hydropower choices via an online and
open access tool. PLoS ONE. 2017;12: e0179393.
doi:\href{https://doi.org/10.1371/journal.pone.0179393}{10.1371/journal.pone.0179393}

\DIFdelbegin %DIFDELCMD < \hypertarget{ref-burgess_2016-HN}{}
%DIFDELCMD < %%%
\DIFdel{37. }\DIFdelend \DIFaddbegin \hypertarget{ref-burgess_2018-HN}{}
\DIFadd{38. }\DIFaddend Burgess MG, Clemence M, McDermott GR, Costello C, Gaines SD. Five
rules for pragmatic blue growth. Marine Policy. \DIFdelbegin \DIFdel{2016;}\DIFdelend \DIFaddbegin \DIFadd{2018;87: 331--339.
}\DIFaddend doi:\href{https://doi.org/10.1016/j.marpol.2016.12.005}{10.1016/j.marpol.2016.12.005}

\hypertarget{ref-nom049sagpesc_2014-V6}{}
\DIFdelbegin \DIFdel{38. }\DIFdelend \DIFaddbegin \DIFadd{39. }\DIFaddend NOM-049-SAG/PESC. NORMA oficial mexicana nom-049-sag/pesc-2014, que
determina el procedimiento para establecer zonas de refugio para los
recursos pesqueros en aguas de jurisdicción federal de los estados
unidos mexicanos. DOF. 2014;

\hypertarget{ref-lgeepa_2017-jL}{}
\DIFdelbegin \DIFdel{39. }\DIFdelend \DIFaddbegin \DIFadd{40. }\DIFaddend LGEEPA. Ley general del equilibrio ecológico y la protección al
ambiente. DOF. 2017; Available:
\url{http://www.diputados.gob.mx/LeyesBiblio/pdf/148/_240117.pdf}

\DIFaddbegin \hypertarget{ref-lester_2017-nh}{}
\DIFadd{41. Lester S, McDonald G, Clemence M, Dougherty D, Szuwalski C. Impacts
of turfs and marine reserves on fisheries and conservation goals:
Theory, empirical evidence, and modeling. BMS. 2017;93: 173--198.
doi:}\href{https://doi.org/10.5343/bms.2015.1083}{10.5343/bms.2015.1083}

\hypertarget{ref-chirico_2017-Rz}{}
\DIFadd{42. Chirico AAD, McClanahan TR, Eklöf JS. Community- and
government-managed marine protected areas increase fish size, biomass
and potential value. PLoS ONE. 2017;12: e0182342.
doi:}\href{https://doi.org/10.1371/journal.pone.0182342}{10.1371/journal.pone.0182342}

\hypertarget{ref-rossetto_2015-V0}{}
\DIFadd{43. Rossetto M, Micheli F, Saenz-Arroyo A, Montes JAE, De Leo GA.
No-take marine reserves can enhance population persistence and support
the fishery of abalone. Can J Fish Aquat Sci. 2015;72: 1503--1517.
doi:}\href{https://doi.org/10.1139/cjfas-2013-0623}{10.1139/cjfas-2013-0623}

\DIFaddend \hypertarget{ref-woodcock_2017-Wm}{}
\DIFdelbegin \DIFdel{40. }\DIFdelend \DIFaddbegin \DIFadd{44. }\DIFaddend Woodcock P, O'Leary BC, Kaiser MJ, Pullin AS. Your evidence or mine?
Systematic evaluation of reviews of marine protected area effectiveness.
Fish Fish. 2017;18: 668--681.
doi:\href{https://doi.org/10.1111/faf.12196}{10.1111/faf.12196}

\DIFaddbegin \hypertarget{ref-carter_2017-Uf}{}
\DIFadd{45. Carter AB, Davies CR, Emslie MJ, Mapstone BD, Russ GR, Tobin AJ, et
al. Reproductive benefits of no-take marine reserves vary with region
for an exploited coral reef fish. Sci Rep. 2017;7: 9693.
doi:}\href{https://doi.org/10.1038/s41598-017-10180-w}{10.1038/s41598-017-10180-w}

\hypertarget{ref-villaseorderbez_website_2017-xE}{}
\DIFadd{46. Villaseñor-Derbez JC, Faro C, Wright M, Martínez J. A guide to
evaluate the effectiveness of no-take marine reserves in mexico
}{[}\DIFadd{Internet}{]}\DIFadd{. 2017. Available:
}\url{https://www.researchgate.net/publication/317840581/_A/_guide/_to/_evaluate/_the/_effectiveness/_of/_no-take/_marine/_reserves/_in/_Mexico}

\DIFaddend \hypertarget{ref-ferraro_2006-oW}{}
\DIFdelbegin \DIFdel{41. }\DIFdelend \DIFaddbegin \DIFadd{47. }\DIFaddend Ferraro PJ, Pattanayak SK. Money for nothing? A call for empirical
evaluation of biodiversity conservation investments. PLoS Biol. 2006;4:
e105.
doi:\href{https://doi.org/10.1371/journal.pbio.0040105}{10.1371/journal.pbio.0040105}

\hypertarget{ref-oecd_website_2017-VV}{}
\DIFdelbegin \DIFdel{42. }\DIFdelend \DIFaddbegin \DIFadd{48. }\DIFaddend OECD. Prices - inflation (cpi) - oecd data {[}Internet{]}. 2017.
Available: \url{https://data.oecd.org/price/inflation-cpi.htm}

\DIFaddbegin \hypertarget{ref-zeileis_2004-7n}{}
\DIFadd{49. Zeileis A. Econometric computing with hc and hac covariance matrix
estimators. J Stat Softw. 2004;11.
doi:}\href{https://doi.org/10.18637/jss.v011.i10}{10.18637/jss.v011.i10}

\DIFaddend \hypertarget{ref-espinosaromero_2014-PY}{}
\DIFdelbegin \DIFdel{43. }\DIFdelend \DIFaddbegin \DIFadd{50. }\DIFaddend Espinosa-Romero MJ, Rodriguez LF, Weaver AH, Villanueva-Aznar C,
Torre J. The changing role of ngos in mexican small-scale fisheries:
From environmental conservation to multi-scale governance. Marine
Policy. 2014;50: 290--299.
doi:\href{https://doi.org/10.1016/j.marpol.2014.07.005}{10.1016/j.marpol.2014.07.005}

\hypertarget{ref-rcore_2017}{}
\DIFdelbegin \DIFdel{44. }\DIFdelend \DIFaddbegin \DIFadd{51. }\DIFaddend R Core Team. R: A language and environment for statistical computing
{[}Internet{]}. Vienna, Austria: R Foundation for Statistical Computing;
2017. Available: \url{https://www.R-project.org/}

\hypertarget{ref-shiny_2017}{}
\DIFdelbegin \DIFdel{45. }\DIFdelend \DIFaddbegin \DIFadd{52. }\DIFaddend Chang W, Cheng J, Allaire J, Xie Y, McPherson J. Shiny: Web
application framework for r {[}Internet{]}. 2017. Available:
\url{https://CRAN.R-project.org/package=shiny}

\hypertarget{ref-shinydashboard_2017}{}
\DIFdelbegin \DIFdel{46. }\DIFdelend \DIFaddbegin \DIFadd{53. }\DIFaddend Chang W, Borges Ribeiro B. Shinydashboard: Create dashboards with
'shiny' {[}Internet{]}. 2017. Available:
\url{https://CRAN.R-project.org/package=shinydashboard}

\hypertarget{ref-rmarkdown_2017}{}
\DIFdelbegin \DIFdel{47. }\DIFdelend \DIFaddbegin \DIFadd{54. }\DIFaddend Allaire J, Cheng J, Xie Y, McPherson J, Chang W, Allen J, et al.
Rmarkdown: Dynamic documents for r {[}Internet{]}. 2017. Available:
\url{https://CRAN.R-project.org/package=rmarkdown}

\hypertarget{ref-knitr_2017}{}
\DIFdelbegin \DIFdel{48. }\DIFdelend \DIFaddbegin \DIFadd{55. }\DIFaddend Xie Y. Knitr: A general-purpose package for dynamic report
generation in r {[}Internet{]}. 2017. Available:
\url{http://yihui.name/knitr/}

\hypertarget{ref-mccay_2017-1m}{}
\DIFdelbegin \DIFdel{49. }\DIFdelend \DIFaddbegin \DIFadd{56. }\DIFaddend McCay B. Territorial use rights in fisheries of the northern pacific
coast of mexico. BMS. 2017;93: 69--81.
doi:\href{https://doi.org/10.5343/bms.2015.1091}{10.5343/bms.2015.1091}

\hypertarget{ref-mccay_2014-CN}{}
\DIFdelbegin \DIFdel{50. }\DIFdelend \DIFaddbegin \DIFadd{57. }\DIFaddend McCay BJ, Micheli F, Ponce-Díaz G, Murray G, Shester G,
Ramirez-Sanchez S, et al. Cooperatives, concessions, and co-management
on the pacific coast of mexico. Marine Policy. 2014;44: 49--59.
doi:\href{https://doi.org/10.1016/j.marpol.2013.08.001}{10.1016/j.marpol.2013.08.001}

\DIFaddbegin \hypertarget{ref-INEGI}{}
\DIFadd{58. Estadistica Geografia e Informatica de Mexico IN de. Marco
geoestadistico nacional. 2017. Available:
}\url{www.inegi.org.mx/geo/contenidos/geoestadistica/m_geoestadistico.aspx}

\hypertarget{ref-tmap_2017}{}
\DIFadd{59. Tennekes M. Tmap: Thematic maps }{[}\DIFadd{Internet}{]}\DIFadd{. 2017. Available:
}\url{https://CRAN.R-project.org/package=tmap}

\DIFaddend \hypertarget{ref-afflerbach_2014-HP}{}
\DIFdelbegin \DIFdel{51. }\DIFdelend \DIFaddbegin \DIFadd{60. }\DIFaddend Afflerbach JC, Lester SE, Dougherty DT, Poon SE. A global survey of
-reserves, territorial use rights for fisheries coupled with marine
reserves. Global Ecology and Conservation. 2014;2: 97--106.
doi:\href{https://doi.org/10.1016/j.gecco.2014.08.001}{10.1016/j.gecco.2014.08.001}

\DIFdelbegin %DIFDELCMD < \hypertarget{ref-lester_2017-nh}{}
%DIFDELCMD < %%%
\DIFdel{52. Lester S, McDonald G, Clemence M, Dougherty D, Szuwalski C. Impacts
of turfs and marine reserves on fisheries and conservation goals:
Theory, empirical evidence, and modeling. BMS. 2017;93: 173--198.
doi:}%DIFDELCMD < \href{https://doi.org/10.5343/bms.2015.1083}{10.5343/bms.2015.1083}
%DIFDELCMD < 

%DIFDELCMD < \hypertarget{ref-rossetto_2015-V0}{}
%DIFDELCMD < %%%
\DIFdel{53. Rossetto M, Micheli F, Saenz-Arroyo A, Montes JAE, De Leo GA, Rochet
M-J. No-take marine reserves can enhance population persistence and
support the fishery of abalone. Can J Fish Aquat Sci. 2015;72:
1503--1517.
doi:}%DIFDELCMD < \href{https://doi.org/10.1139/cjfas-2013-0623}{10.1139/cjfas-2013-0623}
%DIFDELCMD < 

%DIFDELCMD < %%%
\DIFdelend \hypertarget{ref-suman_2010-ez}{}
\DIFdelbegin \DIFdel{54. }\DIFdelend \DIFaddbegin \DIFadd{61. }\DIFaddend Suman CS, Saenz-Arroyo A, Dawson C, Luna MC. Manual de instruccion
de reef check california: Guia de instruccion para el monitoreo del
bosque de sargazo en la peninsula de baja california. Pacific Palisades,
CA, USA: Reef Check Foundation; 2010.

\hypertarget{ref-gutirrez_2011-0U}{}
\DIFdelbegin \DIFdel{55. }\DIFdelend \DIFaddbegin \DIFadd{62. }\DIFaddend Gutiérrez NL, Hilborn R, Defeo O. Leadership, social capital and
incentives promote successful fisheries. Nature. 2011;470: 386--389.
doi:\href{https://doi.org/10.1038/nature09689}{10.1038/nature09689}

\hypertarget{ref-finkbeiner_2015-87}{}
\DIFdelbegin \DIFdel{56. }\DIFdelend \DIFaddbegin \DIFadd{63. }\DIFaddend Finkbeiner EM, Basurto X. Re-defining co-management to facilitate
small-scale fisheries reform: An illustration from northwest mexico.
Marine Policy. 2015;51: 433--441.
doi:\href{https://doi.org/10.1016/j.marpol.2014.10.010}{10.1016/j.marpol.2014.10.010}

\DIFaddbegin \hypertarget{ref-ball_2009-qi}{}
\DIFadd{64. Ball IR, Possingham HP, Watts ME. Marxan and relatives: Software for
spatial conservation prioritization. Spatial conservation prioritization
quantitative methods \& computational tools. United Kingdom: Oxford
University Press; 2009. pp. 185--195. Available:
}\url{https://espace.library.uq.edu.au/view/UQ:200259}

\hypertarget{ref-padleton_2017-vn}{}
\DIFadd{65. Padleton L, Aghmadia G, Browman H, Thurstand R, Kaplan D, Bartolino
V. Debating the effectiveness of marine protected areas. ICES Journal of
Marine Science. 2017;
doi:}\href{https://doi.org/10.1093/icesjms/fsx154}{10.1093/icesjms/fsx154}

\DIFaddend \hypertarget{ref-lundquist_2005-OL}{}
\DIFdelbegin \DIFdel{57. }\DIFdelend \DIFaddbegin \DIFadd{66. }\DIFaddend Lundquist CJ, Granek EF. Strategies for successful marine
conservation: Integrating socioeconomic, political, and scientific
factors. Conserv Biol. 2005;19: 1771--1778.
doi:\href{https://doi.org/10.1111/j.1523-1739.2005.00279.x}{10.1111/j.1523-1739.2005.00279.x}

\section{Supporting information}\label{supporting-information}

\textbf{S1 Appendix. Survey to collect governance information from
fishing communities.} English version

\textbf{S2 Appendix. Survey to collect governance information from
fishing communities.} Spanish version

\textbf{S3 Table. Assigned values and reasoning of socioeconomic and
governance indicators used to color-code the scorecard in MAREA}


\end{document}
